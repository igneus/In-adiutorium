\documentclass[a5paper, twoside]{article}
\usepackage[utf8]{inputenc}

% cestina hazi pri prekladu error a tak zdrzuje a blokuje automaticke generovani;
% pri tvorbe finalniho vystupu je treba ji odkomentovat a preklad spoustet
% rucne, ne pres rake.
% \usepackage{czech} 

\usepackage[left=2cm, right=1.5cm, top=2cm, bottom=1cm]{geometry} % okraje stranky
\usepackage[colorlinks=true, citecolor=black, linkcolor=black, urlcolor=black]{hyperref} % hypertextove odkazy
\usepackage{datetime} % formaty data
\usepackage{multicol} % sazba ve sloupcich
\usepackage{lettrine} % inicialky
\usepackage{fancyhdr} % hlavicky stranek
\usepackage{color} % barevny text

\usepackage[T1]{fontenc}
% font Palatino:
% \usepackage[sc]{mathpazo}
% \linespread{1.05}        

% font Bookman:
\usepackage{bookman}

% Definice prikazu uzivanych ve vsech svazcich antifonare

% Predpokladane balicky: 
% datetime, hyperref

%
% spolecna nastaveni
%

\setlength{\parindent}{0.4cm}

%
% Titulni strana a tiraz svazku antifonare.
% prikazy vyuzivaji prikazy, ktere musi byt
% definovany v dokumentu:
% \soubornyNazev, \edicniRada, \nazev, \cisloSvazku, \autor, \textDoTiraze
%

\newcommand{\titulniStrankaSvazkuAntifonare}{
  \pagestyle{empty} % bez hlavicky a bez cisla stranky
  \begin{titlepage}
    \begin{center}
      \soubornyNazev

      \vspace{0.5cm}

      \edicniRada
      
      svazek \cisloSvazku
      
      \vspace*{6cm}
      
      {\Huge \textbf{\nazev}}
          
      \vspace{1cm}
      
      {\large \autor}
      
      \vfill
      In adiutorium
      
      \onlyyeardate \today
    \end{center}
  \end{titlepage}
}

\newcommand{\patitulSvazkuAntifonare}{
  \begin{center}
    \soubornyNazev

    \nazev
  \end{center}

  \vfill
}

\newcommand{\frontispisSvazkuAntifonare}{
  \mbox{}
}

\newcommand{\tirazProjekt}{
  projekt In adiutorium - hudba k české liturgii hodin\\  
    \url{http://inadiutorium.xf.cz}
}

\newcommand{\lilypondbookVersion}{
  \input{|"lilypond-book --version"}
}

\newcommand{\tirazSazbaProgramem}{
  Sazba programy \LaTeX\mbox{} a LilyPond % \lilypondbookVersion
}

\newcommand{\tirazSvazkuAntifonare}{
  \pagestyle{empty} % bez hlavicky a bez cisla stranky
  \setlength{\parskip}{0.6cm}

  \begin{center}
    \soubornyNazev\\
    \edicniRada, svazek \cisloSvazku
    
    {\Large \textbf{\nazev}}

    {\large \autor}

    \tirazProjekt
    
    \tirazSazbaProgramem
  \end{center}

  \setlength{\parindent}{0cm} % prvni radek odstavce neodsazovat
  \textDoTiraze

  \begin{center}
    \dmyyyydate \today
  \end{center}
}

%
% Standartni licencni poznamka pouzitelna pro vetsinu svazku
% Antifonare k Denni modlitbe cirkve
%

\newcommand{\licencniPoznamka}{%
  Používání \emph{nápěvů} je vázáno licencí 
  \href{http://creativecommons.org/licenses/by-sa/3.0/deed.cs}{Creative Commons 
    At\-tri\-bu\-tion-\-Sha\-re-A\-li\-ke 3.0 Unported},
  která dává komukoli právo je dále šířit, upravovat a využívat
  ve svých vlastních dílech, za předpokladu, že uvede informaci
  o autorovi původního díla a při dalším šíření zachová původní licenční 
  podmínky.
  (Výjimku představují nápěvy převzaté z jiných publikací - je to vždy
  výslovně uvedeno. Tyto nápěvy vesměs patří k prastarému pokladu
  gregoriánského chorálu a jsou tedy public domain.)
  
  \emph{Texty Denní modlitby církve} 
  jsou majetkem České biskupské konference. 
  Výše uvedenou licenci proto nelze vztahovat na antifonář jako celek -
  týká se pouze jeho hudební složky.

  Texty žalmů a kantik jsou převzaté z webu \url{http://ebreviar.cz},
  jehož tvůrcům za jejich práci tímto srdečně děkuji.
  Z textů je vypuštěno to, co se zdá být pro modlitbu v chóru
  nadbytečné nebo rušivé (číslování žalmů i podle Vulgáty,
  nadpisy žalmů a připojené citáty z Písma a Otců)
  a kde jsem shledal odchylky od vydání breviáře z r. 1994, jsou
  podle něj opraveny.
}

%
% Vyrobi prazdnou stranku
%

\newcommand{\prazdnaStranka}{
  \newpage \mbox{}
  \newpage
}

%
% Nasledujici prikazy vyrabeji hlavicku a paticku pro
% male materialy obsahem a razem podobne materialum
% z LilyPondu
%

\newcommand{\malyTitulek}{
  \begin{center}
    {\large \textbf{\nazev}}
  \end{center}
  \begin{flushright}
    \autor
  \end{flushright}
}

\newcommand{\malaTiraz}{
  datum: \dmyyyydate \today
  
  licence: Creative Commons Attribution-ShareAlike 3.0 Unported
  
  projekt: In adiutorium - noty k liturgii hodin (http://inadiutorium.xf.cz)
  
  sazba programy \LaTeX a LilyPond \lilypondbookVersion
}

% cesky format data:
\renewcommand{\dateseparator}{.}

% format data, kde je jen rok
\newdateformat{onlyyeardate}{%
\THEYEAR}


%
% Nasledujici prikazy slouzi strukturovani uvnitr textu antifonare
%

% tyden
\newcommand{\nadpisTyden}[2]{
  \phantomsection%
  \begin{center}
  {\LARGE \textsc{#1}}
  \end{center}
  \fancyhead[CE]{#2}
  \addcontentsline{toc}{section}{#2}
}

% den
\newcommand{\nadpisDen}[1]{
  \phantomsection%
  \begin{center}
  {\LARGE #1}
  \end{center}
  \fancyhead[CO]{#1}
  \addcontentsline{toc}{subsection}{#1}
}

% hodinka
\newcommand{\nadpisHodinka}[1]{
  \begin{center}
  \textbf{#1}
  \end{center}
  \nopagebreak
}

% invitatorium samozrejme neni samostatna hodinka,
% ale tim, ze muze byt soucasti dvou ruznych,
% se funkcne osamostatnilo a davame mu nadpis na urovni hodinky
\newcommand{\invitatorium}{
  \nadpisHodinka{Invitatorium}
}

\newcommand{\modlitbaSeCtenim}{
  \nadpisHodinka{Modlitba se čtením}
}

\newcommand{\ranniChvaly}{
  \nadpisHodinka{Ranní chvály}
}

\newcommand{\modlitbaUprostredDne}{
  \nadpisHodinka{Modlitba uprostřed dne}
}

\newcommand{\nespory}{
  \nadpisHodinka{Nešpory}
}

\newcommand{\nesporyI}{
  \nadpisHodinka{První nešpory}
}

\newcommand{\nesporyII}{
  \nadpisHodinka{Druhé nešpory}
}

\newcommand{\kompletar}{
  \nadpisHodinka{Kompletář}
}

% nadpis hodinky uzivany v indexu svatku svatecniho zaltare
\newcommand{\nadpisHodinkaVIndexu}[1]{
  \noindent \underline{#1:}
}

% Varianty tech samych prikazu pro index svatku:
\newcommand{\idxModlitbaSeCtenim}{
  \nadpisHodinkaVIndexu{MČ}
}

\newcommand{\idxVigilie}{
  \nadpisHodinkaVIndexu{VI}
}

\newcommand{\idxRanniChvaly}{
  \nadpisHodinkaVIndexu{RCH}
}

\newcommand{\idxModlitbaUprostredDne}{
  \nadpisHodinkaVIndexu{MU}
}

\newcommand{\idxNespory}{
  \nadpisHodinkaVIndexu{N}
}

\newcommand{\idxNesporyI}{
  \nadpisHodinkaVIndexu{1N}
}

\newcommand{\idxNesporyII}{
  \nadpisHodinkaVIndexu{2N}
}

\newcommand{\idxKompletar}{
  \nadpisHodinkaVIndexu{K}
}

\newenvironment{idxObsahHory}{}{}


% Cilem je, aby tento nadpis nad sebou nemel tolik mista
% a tak umoznoval "prilepeni" predchazejici rubriky.
\newcommand{\nadpisZalmuBezMezery}[1]{%

  % prostredi center si svevolne pridava
  % dodatecny vertikalni prostor - musime ho tedy umele ubrat:
  \vspace{-5mm}%
  \begin{center}%
    \textbf{#1}%
  \end{center}%
  \nopagebreak%
}

% obvykle napr. "zalm 94"
\newcommand{\nadpisZalmu}[1]{%
  \vspace{2mm}%
  \begin{center}%
    \textbf{#1}%
  \end{center}%
  \nopagebreak%
}

\newcommand{\titulusPsalmi}[1]{\nadpisZalmu{#1}}


% Napr. "hymnus", "responsorium", "zalmy"
\newcommand{\nadpisTypTextu}[1]{
  \begin{flushleft}
  \noindent
  \textsc{#1}
  \end{flushleft}
  \nopagebreak
}

% rubrikovany text - treba i pro vlozeni do nerubrikovaneho
\newcommand{\rubr}[1]{%
  \textit{#1}%
}

% rubrika - samostatna
\newcommand{\rubrika}[1]{
  \vspace{2mm}
  \rubr{#1}
  \vspace{2mm}
}

% mezera, kterou se oddeli rubrika od textu, ke kteremu se _nevztahuje_,
% aby byla vic prilepena k tomu, ke kteremu se vztahuje
\newcommand{\rubrikaMezera}{
  \vspace{4mm}
}

% rubrika - predchazejici
\newcommand{\rubrikaPred}[1]{
  \rubrikaMezera
  % \textit{#1}
  \rubr{#1}
  \vspace{1mm}
}

% rubrika - nasledujici
\newcommand{\rubrikaPo}[1]{
  \vspace{1mm}
  % \textit{#1}
  \rubr{#1}
  \rubrikaMezera
}

% mezera pred hvezdickou pulici vers zalmu muze byt v pripade potreby
% zcela zkomprimovana
\newcommand{\asterisk}{%
\nobreak\hskip \fontdimen1\font plus \fontdimen2\font minus \fontdimen1\font
* }

\newcommand{\flexa}{%
\nobreak\hskip \fontdimen1\font plus \fontdimen2\font minus \fontdimen1\font
\dag\mbox{} }

\newcommand{\znackaStrofaZalmu}{ 
  \rubr{--}
}

\newcommand{\doxologieZkratka}{%
  \nopagebreak%
  \hspace{\fill}\mbox{}\linebreak[0]\hspace*{\parindent}\mbox{Sláva Otci.}%
}

% horizontalni cara dlouha jako pul stranky siroky sloupec sloupec
% ve svazku antifonare
\newcommand{\oddelovac}{
\noindent \line(1,0){150}
}


% z logiky liturgickych knih by znacky pro versiky mely byt
% rubrikovane, ne tucne. Zkousel jsem to, to ale v okoli zanikaji
% a myslim ze je dobre je vyrazne odlisit od psalmodie, ktera je alespon
% v zaltari bez not obklopuje
\newcommand{\versikSamostatny}[2]{
 
  \vspace{3mm}
  \noindent \textbf{V.} #1\\
  \textbf{O.} #2
}
\newcommand{\versikCteni}[2]{
  \versikSamostatny{#1}{#2}
}
\newcommand{\versik}[3]{

  \vspace{1mm}
  \rubrika{#1}\\
  \noindent \textbf{V.} #2\\
  \textbf{O.} #3
}
\newcommand{\versikTercie}[2]{\versik{dopoledne:}{#1}{#2}}
\newcommand{\versikSexta}[2]{\versik{v poledne:}{#1}{#2}}
\newcommand{\versikNona}[2]{\versik{odpoledne:}{#1}{#2}}

%% prikazy predpokladane preprocesorem zalmu pslm.rb:

% vetsina notovanych svazku chce mit kazdy zalm ve zvlastnim
% dvousloupci; v tech ostatnich se to predefinuje podle potreby
\newenvironment{psalmus}{\begin{multicols}{2}}{\end{multicols}}

\newcommand{\flex}{\flexa}


% v tomto svazecku se nadpis hodinky pouziva pouze v indexu
% na konci knihy a tak je zadouci jiny vzhled -
% predevsim zarovnani doleva
\renewcommand{\nadpisHodinka}[1]{
  \noindent\textbf{#1}:
  \newline
}

\newcommand{\autor}{}
\newcommand{\soubornyNazev}{Antifonář k Denní modlitbě církve}
\newcommand{\cisloSvazku}{5}
\newcommand{\nazev}{Žaltář pro slavnosti a svátky}

\renewcommand{\tirazSazbaProgramem}{
  Vysázeno programem \LaTeX.
}
\newcommand{\textDoTiraze}{
  Texty Denní modlitby církve jsou majetkem České biskupské konference. 
  
  Texty žalmů a kantik jsou převzaté z webu \url{http://ebreviar.cz},
  jehož tvůrcům za jejich práci tímto srdečně děkuji.
  Z textů je vypuštěno to, co se zdá být pro modlitbu v chóru
  nadbytečné nebo rušivé (číslování žalmů i podle Vulgáty,
  nadpisy žalmů a připojené citáty z Písma a Otců)
  a kde jsem shledal odchylky od vydání breviáře z r. 1994, jsou
  podle něj opraveny.
  
  Text žalmů a kantik značkami opatřil a k sazbě připravil Jakub Pavlík.
}

\newenvironment{hora}{
  \begin{multicols}{2}
}{
  \end{multicols}
}

\newcommand{\rubrikaZalmBylvInvitatoriu}{
  \rubrika{Pokud se následující žalm zpíval jako invitatorium,
  nahradí se zde žalmem 95, str. \pageref{zalm24}.}
}

% Vytvori pro text (zalm/kantikum) label a polozku v obsahu
% label, citelny nazev
\newcommand{\labelText}[2]{
  \label{#1}
  \addcontentsline{toc}{subsection}{#2}
}

% Staci jediny argument - cislo zalmu
\newcommand{\labelZalm}[1]{
  \labelText{z#1}{Žalm #1}
  }

\newcommand{\zalmRef}[1]{
  Žalm #1, str. \pageref{z#1}}

\newcommand{\textRef}[2]{
  #2, str. \pageref{#1}}

\begin{document}

\pagestyle{empty}

\titulniStrankaSvazkuAntifonare

\prazdnaStranka

% Nastaveni sazby do sloupcu:
\setlength{\columnseprule}{1pt} % cara oddelujici sloupce
\setlength{\columnsep}{20pt} % prostor mezi sloupci

\fancyhead{}
\fancyhead[LE,RO]{\thepage}
\fancyfoot{}

\pagestyle{fancy}

\section{Úvod}

Tento žaltář je určen pro dny, kdy se v breviáři nepoužívají žalmy
z cyklu žaltáře, ale ze společných textů o svatých nebo z vlastních textů 
slavnosti nebo svátku.

V zájmu úspory místa a za cenu jistého zmenšení uživatelského pohodlí
nejsou žalmy a kantika řazeny jako v breviáři a neopakují se,
nýbrž jsou v příslušných oddílech řazeny podle pořadí v katolickém
biblickém kánonu. Na konci knihy je přehledný seznam všech slavností
a svátků, které mají vlastní žalmy, spolu s odkazy na texty.

Doporučuji opatřit výtisky dostatečným množstvím záložek (ideální se
zdají být 3-4).

\section{Kantika z evangelií}

\begin{hora}
\labelText{ben}{Zachariášovo kantikum}
\nadpisTypTextu{Zachariášovo kantikum (Benedictus)}
\input{generovane/zaltar/kantikum_benedictus.tex}
\columnbreak
\labelText{mag}{kantikum Panny Marie}
\nadpisTypTextu{Kantikum Panny Marie (Magnificat)}
\input{generovane/zaltar/kantikum_magnificat.tex}
\end{hora}

\section{Žalmy invitatoria}

\begin{hora}
\labelZalm{95}
\input{generovane/svatecnizaltar/zalm95.tex}
\labelZalm{100}
\input{generovane/svatecnizaltar/zalm100.tex}
\labelZalm{67}
\input{generovane/svatecnizaltar/zalm67.tex}
\labelZalm{24}
\input{generovane/svatecnizaltar/zalm24.tex}
\end{hora}

\clearpage

\section{Často užívané celky}

\subsection{Neděle 1. týdne - ranní chvály}

\ranniChvaly
\begin{hora}
\label{zalmyne1trch}
\input{generovane/svatecnizaltar/zalm63.tex}
\input{generovane/svatecnizaltar/kantikum_dan3i.tex}
\input{generovane/svatecnizaltar/zalm149.tex}
\end{hora}

\subsection{Doplňovací cyklus žalmů pro modlitbu uprostřed dne}{Doplňovací cyklus žalmů}

\label{zalmydoplncyklus}
\nadpisHodinka{První oddíl (dopoledne)}
\begin{hora}
\input{generovane/svatecnizaltar/zalm120.tex}
\input{generovane/svatecnizaltar/zalm121.tex}
\input{generovane/svatecnizaltar/zalm122.tex}
\end{hora}

\nadpisHodinka{Druhý oddíl (v poledne)}
\begin{hora}
\input{generovane/svatecnizaltar/zalm123.tex}
\input{generovane/svatecnizaltar/zalm124.tex}
\input{generovane/svatecnizaltar/zalm125.tex}
\end{hora}

\nadpisHodinka{Třetí oddíl (odpoledne)}
\begin{hora}
\input{generovane/svatecnizaltar/zalm126.tex}
\input{generovane/svatecnizaltar/zalm127.tex}
\input{generovane/svatecnizaltar/zalm128.tex}
\end{hora}

\section{Žalmy}

\begin{hora}
% soubor generovany skriptem listofpsalms.rb:
\input{generovane/svatecnizaltar/svatecnizaltar_index.txt.psalms.tex}
\end{hora}

\section{Starozákonní kantika}

\begin{hora}
\labelText{kantiz38}{Iz 38}
\input{generovane/svatecnizaltar/kantikum_iz38.tex}
\labelText{kanthab3}{Hab 3}
\input{generovane/svatecnizaltar/kantikum_hab3.tex}
\end{hora}

\section{Novozákonní kantika}

\begin{hora}
\labelText{kantef1}{Ef 1}
\input{generovane/svatecnizaltar/kantikum_ef1.tex}
\labelText{kantfp2}{Flp 2}
\input{generovane/svatecnizaltar/kantikum_fp2.tex}
\labelText{kantkol1}{Kol 1}
\input{generovane/svatecnizaltar/kantikum_kol1.tex}
\labelText{kant1tim3}{1 Tim 3}
\nadpisZalmu{kantikum\\ srov. 1 Tim 3, 16}

\rubr{O.} Chvalte Pána, všechny národy.

On přišel v lidské přirozenosti,~*
byl ospravedlněn Duchem.

\rubr{O.} Chvalte Pána, všechny národy.

Ukázal se andělům,~*
byl hlásán pohanům.

\rubr{O.} Chvalte Pána, všechny národy.

Došel víry ve světě,~*
byl vzat do slávy.

\rubr{O.} Chvalte Pána, všechny národy. 
\labelText{kant1petr2}{1 Petr 2}
\input{generovane/svatecnizaltar/kantikum_1petr2.tex}
\labelText{kantzj4}{Zj 4}
\input{generovane/svatecnizaltar/kantikum_zj4.tex}
\labelText{kantzj11}{Zj 11}
\input{generovane/svatecnizaltar/kantikum_zj11.tex}
\labelText{kantzj15}{Zj 15}
\input{generovane/svatecnizaltar/kantikum_zj15.tex}
\labelText{kantzj19}{Zj 19}
\nadpisZalmu{Srov. Zj 19, 1-7}

Aleluja.
Vítězství, sláva a moc našemu Bohu, 

O. aleluja.

neboť jeho soudy jsou pravdivé a spravedlivé. 

O. Aleluja, aleluja.\\

Aleluja. 
Chvalte našeho Boha, všichni, kdo mu sloužíte 

O. aleluja.

a kdo se ho bojíte, malí i velcí! 

O. Aleluja, aleluja.\\

Aleluja. 
Pán, náš Bůh vševládný, se ujal království! 

O. aleluja.

Radujme se, jásejme a vzdejme mu čest! 

O. Aleluja, aleluja.\\

Aleluja. 
Neboť nadešla Beránkova svatba, 

O. aleluja.

jeho nevěsta se připravila. 

O. Aleluja, aleluja. 

\end{hora}

\section{Index svátků}

% generovany soubor
\input{generovane/svatecnizaltar/svatecnizaltar_index.txt.index.tex}

\clearpage
\pagestyle{empty} % bez hlavicky a bez cisla stranky
\tableofcontents

\clearpage
\tirazSvazkuAntifonare

\end{document}