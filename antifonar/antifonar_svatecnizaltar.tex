\documentclass[a5paper, twoside]{article}
\usepackage[utf8]{inputenc}

\usepackage[czech]{babel} 

\usepackage[left=2cm, right=1.5cm, top=2cm, bottom=1cm]{geometry} % okraje stranky
\usepackage[colorlinks=true, citecolor=black, linkcolor=black, urlcolor=black]{hyperref} % hypertextove odkazy
\usepackage{datetime} % formaty data
\usepackage{multicol} % sazba ve sloupcich
\usepackage{lettrine} % inicialky
\usepackage{fancyhdr} % hlavicky stranek
\usepackage{color} % barevny text

% potreba pro znacky.tex:
\usepackage{zref-savepos}
\usepackage{changepage}
\usepackage{etoolbox}

\usepackage[T1]{fontenc}
% font Bookman:
\usepackage{bookman}

% Definice prikazu uzivanych ve vsech svazcich antifonare

% Predpokladane balicky: 
% datetime, hyperref


%
% Titulni strana a tiraz svazku antifonare.
% prikazy vyuzivaji prikazy, ktere musi byt
% definovany v dokumentu:
% \soubornyNazev, \nazev, \cisloSvazku, \autor, \textDoTiraze
%

\newcommand{\titulniStrankaSvazkuAntifonare}{
  \pagestyle{empty} % bez hlavicky a bez cisla stranky
  \begin{titlepage}
    \begin{center}
      \soubornyNazev
      
      Svazeček \cisloSvazku.
      
      \vspace*{7cm}
      
      {\Huge \textbf{\nazev}}
          
      \vspace{1cm}
      
      {\large \autor}
      
      \vfill
      In adiutorium
      
      \onlyyeardate \today
    \end{center}
  \end{titlepage}
}

\newcommand{\tirazProjekt}{
  projekt In adiutorium - hudba k české liturgii hodin\\  
    \url{http://inadiutorium.xf.cz}
}

\newcommand{\lilypondbookVersion}{
  \input{|"lilypond-book --version"}
}

\newcommand{\tirazSazbaProgramem}{
  Vysázeno programy \LaTeX  a LilyPond % \lilypondbookVersion
}

\newcommand{\tirazSvazkuAntifonare}{
  \pagestyle{empty} % bez hlavicky a bez cisla stranky
  \setlength{\parskip}{0.6cm}

  \begin{center}
    \soubornyNazev, svazeček \cisloSvazku.
    
    {\Large \textbf{\nazev}}

    {\large \autor}

    \tirazProjekt
    
    \tirazSazbaProgramem
  \end{center}

  \setlength{\parindent}{0cm} % prvni radek odstavce neodsazovat
  \textDoTiraze

  \begin{center}
    \dmyyyydate \today
  \end{center}
}

%
% Vyrobi prazdnou stranku
%

\newcommand{\prazdnaStranka}{
  \newpage \mbox{}
  \newpage
}

%
% Nasledujici prikazy vyrabeji hlavicku a paticku pro
% male materialy obsahem a razem podobne materialum
% z LilyPondu
%

\newcommand{\malyTitulek}{
  \begin{center}
    {\large \textbf{\nazev}}
  \end{center}
  \begin{flushright}
    \autor
  \end{flushright}
}

\newcommand{\malaTiraz}{
  datum: \dmyyyydate \today
  
  licence: Creative Commons Attribution-ShareAlike 3.0 Unported
  
  projekt: In adiutorium - noty k liturgii hodin (http://inadiutorium.xf.cz)
  
  sazba programy \LaTeX a LilyPond \lilypondbookVersion
}

% cesky format data:
\renewcommand{\dateseparator}{.}

% format data, kde je jen rok
\newdateformat{onlyyeardate}{%
\THEYEAR}


%
% Nasledujici prikazy slouzi strukturovani uvnitr textu antifonare
%

% tyden
\newcommand{\nadpisTyden}[2]{
  \begin{center}
  {\LARGE \textsc{#1}}
  \end{center}
  \fancyhead[CE]{#2}
  \addcontentsline{toc}{section}{#2}
}

% den
\newcommand{\nadpisDen}[1]{
  \begin{center}
  {\LARGE #1}
  \end{center}
  \fancyhead[CO]{#1}
  \addcontentsline{toc}{subsection}{#1}
}

% hodinka
\newcommand{\nadpisHodinka}[1]{
  \begin{center}
  \textbf{#1}
  \end{center}
}

\newcommand{\modlitbaSeCtenim}{
  \nadpisHodinka{Modlitba se čtením}
}

\newcommand{\ranniChvaly}{
  \nadpisHodinka{Ranní chvály}
}

\newcommand{\modlitbaUprostredDne}{
  \nadpisHodinka{Modlitba uprostřed dne}
}

\newcommand{\nespory}{
  \nadpisHodinka{Nešpory}
}

\newcommand{\nesporyI}{
  \nadpisHodinka{První nešpory}
}

\newcommand{\nesporyII}{
  \nadpisHodinka{Druhé nešpory}
}

\newcommand{\kompletar}{
  \nadpisHodinka{Kompletář}
}

% nadpis hodinky uzivany v indexu svatku svatecniho zaltare
\newcommand{\nadpisHodinkaVIndexu}[1]{
  \noindent#1:
  \newline\indent
}

% Varianty tech samych prikazu pro index svatku:
\newcommand{\idxModlitbaSeCtenim}{
  \nadpisHodinkaVIndexu{modlitba se čtením}
}

\newcommand{\idxRanniChvaly}{
  \nadpisHodinkaVIndexu{ranní chvály}
}

\newcommand{\idxModlitbaUprostredDne}{
  \nadpisHodinkaVIndexu{modlitba uprostřed dne}
}

\newcommand{\idxNespory}{
  \nadpisHodinkaVIndexu{nešpory}
}

\newcommand{\idxNesporyI}{
  \nadpisHodinkaVIndexu{první nešpory}
}

\newcommand{\idxNesporyII}{
  \nadpisHodinkaVIndexu{druhé nešpory}
}

\newcommand{\idxKompletar}{
  \nadpisHodinkaVIndexu{kompletář}
}


% Cilem je, aby tento nadpis nad sebou nemel tolik mista
% a tak umoznoval "prilepeni" predchazejici rubriky.
\newcommand{\nadpisZalmuBezMezery}[1]{

  % Nepouziva prostredi center, protoze to si svevolne pridava
  % dodatecny vertikalni prostor
  \begingroup\centering 
    \textbf{#1}
  \endgroup
  \vspace{3mm}
  \nopagebreak
}

% obvykle napr. "zalm 94"
\newcommand{\nadpisZalmu}[1]{
  \vspace{2mm}
  \begin{center}
    \textbf{#1}
  \end{center}
  \nopagebreak
}


% Napr. "hymnus", "responsorium", "zalmy"
\newcommand{\nadpisTypTextu}[1]{
  \begin{flushleft}
  \noindent
  \rubr{\textsc{#1}}
  \end{flushleft}
  \nopagebreak
}

% rubrikovany text - treba i pro vlozeni do nerubrikovaneho
\newcommand{\rubr}[1]{
  \textcolor{red}{#1}
}

% rubrika - samostatna
\newcommand{\rubrika}[1]{
  \vspace{2mm}
  % \textit{#1}
  \rubr{#1}
  \vspace{2mm}
}

% mezera, kterou se oddeli rubrika od textu, ke kteremu se _nevztahuje_,
% aby byla vic prilepena k tomu, ke kteremu se vztahuje
\newcommand{\rubrikaMezera}{
  \vspace{4mm}
}

% rubrika - predchazejici
\newcommand{\rubrikaPred}[1]{
  \rubrikaMezera
  % \textit{#1}
  \rubr{#1}
  \vspace{1mm}
}

% rubrika - nasledujici
\newcommand{\rubrikaPo}[1]{
  \vspace{1mm}
  % \textit{#1}
  \rubr{#1}
  \rubrikaMezera
}

\newcommand{\znackaStrofaZalmu}{ 
  \rubr{--}
}

% znacky.tex
%
% potrebuje baliky: zref-savepos, changepage
%
% Umoznuje ve dvousloupcovem usporadani stranky umistit na okraj
% (pravy nebo levy, podle sloupce) varovnou znacku.

% Nastaveni dulezita pro spravne fungovani:
% -----------------------------------
\strictpagecheck % pro makra tvorici vystrazne znacky

\newcounter{zalmVersUpozorneni}
\setcounter{zalmVersUpozorneni}{1}
% ------------------------------------

% konstanty uzivane v nasledujicich makrech
\def\pulstranky{13773045}%sp; vypocteno; pul stranky A5
\def\vzdalenostZnackyOdTextu{0.5}%cm

% argument: label ulozene pozice. Vrati hodnotu pro \hskip na levy okr. stranky
\def\kLevemuOkraji#1{%
  -\zposx{#1}%
}

% Macro written for me by Yiannis Lazarides
% http://tex.stackexchange.com/questions/50216/margin-notes-in-a-multicolumn-environment
% I modified it slightly for my needs.
%
% uses package zref-savepos
%
% Arguments: 
% #1 unique label identifier
% #2 text
%
% In the two-columned setup it puts given text besides the column
% where the macro is called
%
% Numbers which depend on the page's geometry!!!!!!!:
% 2.0cm & 1.5cm: margins of the page (I don't want to have them hardcoded,
% but I haven't found any way to determine them at runtime.)
\def\putmarginpar#1#2{%
  \zsavepos{#1}%
  \checkoddpage
  \ifnum\pulstranky>\number\zposx{#1}%
    % tady je dimen100 levy okraj stranek
    \ifoddpage% liche - siroky
      \dimen100=2.0cm%
    \else%
      \dimen100=1.5cm%
    \fi%
    \hbox to 0pt{\hskip\dimexpr\kLevemuOkraji{#1}sp + \dimen100 - \vzdalenostZnackyOdTextu cm \relax#2}%
  \else
    % tady je dimen100 pravy okraj stranek
    \ifoddpage% liche - uzky
      \dimen100=1.5cm%
    \else%
      \dimen100=2cm%
    \fi%
     \hbox to 0pt{\hskip\dimexpr\kLevemuOkraji{#1}sp +\pulstranky sp * 2 - \dimen100 + \vzdalenostZnackyOdTextu cm \relax#2}%
  \fi%
}

%
% Znacka: varovny vykricnik na okraji stranky vedle aktualni radky
% (Prikaz predpoklada dvousloupcove usporadani, v jednosloupcovem bude
% pracovat nelogicky)
%
\newcommand{\zalmVersUpozorneni}{
  \stepcounter{zalmVersUpozorneni}
  \putmarginpar{znac:\arabic{zalmVersUpozorneni}}{\textup{\textbf{!}}}
}


% !!!! KE ZLOMU: 24.3.2012 jsem musel pri zlomu zasahovat i do automaticky
% generovanych souboru (zalmy; dokonce i do textu jednoho kantika!).
% Kniha tedy obsahuje nektere zlomove znacky, ale ne vsechny. !!!

\newcommand{\autor}{}
\newcommand{\soubornyNazev}{Antifonář k Denní modlitbě církve}
\newcommand{\cisloSvazku}{5}
\newcommand{\nazev}{Žaltář pro slavnosti a svátky}

\renewcommand{\tirazSazbaProgramem}{
  Vysázeno programem \LaTeX.
}
\newcommand{\textDoTiraze}{
  Texty Denní modlitby církve jsou majetkem České biskupské konference. 
  
  Texty žalmů a kantik jsou převzaté z webu \url{http://ebreviar.cz},
  jehož tvůrcům za jejich práci tímto srdečně děkuji.
  Z textů je vypuštěno to, co se zdá být pro modlitbu v chóru
  nadbytečné nebo rušivé (číslování žalmů i podle Vulgáty,
  nadpisy žalmů a připojené citáty z Písma a Otců)
  a kde jsem shledal odchylky od vydání breviáře z r. 1994, jsou
  podle něj opraveny.
  
  Text žalmů a kantik značkami opatřil a k sazbě připravil Jakub Pavlík.
}

\newenvironment{hora}{
  \begin{multicols}{2}
}{
  \end{multicols}
}

\newenvironment{psalmus}{}{}

\newcommand{\rubrikaZalmBylvInvitatoriu}{
  \rubrika{Pokud se následující žalm zpíval jako invitatorium,
  nahradí se zde žalmem 95, str. \pageref{zalm24}.}
}

\newcommand{\laudyNedelePrvnihoTydne}{
  \mbox{Žalmy nedělní z 1. týdne, str. \pageref{zalmyne1trch}}}

% Vytvori pro text (zalm/kantikum) label a polozku v obsahu
% label, citelny nazev
\newcommand{\labelText}[2]{
  \label{#1}
  \addcontentsline{toc}{subsection}{#2}
}

% Staci jediny argument - cislo zalmu
\newcommand{\labelZalm}[1]{
  \labelText{z#1}{Žalm #1}
  }

\newcommand{\zalmRef}[1]{
  Žalm #1, str. \pageref{z#1}}

\newcommand{\textRef}[2]{
  #2, str. \pageref{#1}}

\begin{document}

\pagestyle{empty}

\titulniStrankaSvazkuAntifonare

\prazdnaStranka

% Nastaveni sazby do sloupcu:
\setlength{\columnseprule}{1pt} % cara oddelujici sloupce
\setlength{\columnsep}{20pt} % prostor mezi sloupci

\fancyhead{}
\fancyhead[LE,RO]{\thepage}
\fancyfoot{}

\pagestyle{fancy}

\section{Kantika z evangelií}
\fancyhead[CE,CO]{Kantika z evangelií}

\begin{hora}
\labelText{ben}{Zachariášovo kantikum}
\input{generovane/zaltar/kantikum_benedictus.tex}
\columnbreak
\labelText{mag}{kantikum Panny Marie}
\input{generovane/zaltar/kantikum_magnificat.tex}
\end{hora}

\clearpage
\section{Žalmy invitatoria}
\fancyhead[CE,CO]{Invitatorium}

\begin{hora}
\labelZalm{95}
\input{generovane/svatecnizaltar/zalm95.tex}
\columnbreak % ZLOM1
\labelZalm{100}
\input{generovane/svatecnizaltar/zalm100.tex}
\labelZalm{67}
\input{generovane/svatecnizaltar/zalm67.tex}
\labelZalm{24}
\input{generovane/svatecnizaltar/zalm24.tex}
\end{hora}

\clearpage
\section{Často užívané celky}

\subsection{Neděle 1. týdne - ranní chvály}
\fancyhead[CE,CO]{Ranní chvály neděle 1. týdne}

\ranniChvaly
\begin{hora}
\label{zalmyne1trch}
\input{generovane/svatecnizaltar/zalm63.tex}
\input{generovane/svatecnizaltar/kantikum_dan3iii.tex}
\input{generovane/svatecnizaltar/zalm149.tex}
\end{hora}

\subsection{Doplňovací cyklus žalmů pro modlitbu uprostřed dne}
\fancyhead[CE,CO]{Doplňovací cyklus}

\label{zalmydoplncyklus}
\nadpisHodinka{První oddíl (dopoledne)}
\begin{hora}
\input{generovane/svatecnizaltar/zalm120.tex}
\input{generovane/svatecnizaltar/zalm121.tex}
\input{generovane/svatecnizaltar/zalm122.tex}
\end{hora}

\nadpisHodinka{Druhý oddíl (v poledne)}
\begin{hora}
\input{generovane/svatecnizaltar/zalm123.tex}
\input{generovane/svatecnizaltar/zalm124.tex}
\input{generovane/svatecnizaltar/zalm125.tex}
\end{hora}

\nadpisHodinka{Třetí oddíl (odpoledne)}
\begin{hora}
\input{generovane/svatecnizaltar/zalm126.tex}
\input{generovane/svatecnizaltar/zalm127.tex}
\input{generovane/svatecnizaltar/zalm128.tex}
\end{hora}

\clearpage
\section{Žalmy}
\fancyhead[CE,CO]{Žalmy}

\begin{hora}
% soubor generovany skriptem listofpsalms.rb:
\input{generovane/svatecnizaltar/svatecnizaltar_index.txt.psalms.tex}
\end{hora}

\clearpage
\section{Starozákonní kantika}
\fancyhead[CE,CO]{Starozákonní kantika}

\begin{hora}
\labelText{kantiz38}{Iz 38}
\input{generovane/svatecnizaltar/kantikum_iz38.tex}
\columnbreak % ZLOM3
\labelText{kanthab3}{Hab 3}
\input{generovane/svatecnizaltar/kantikum_hab3.tex}
\end{hora}

\clearpage
\section{Novozákonní kantika}
\fancyhead[CE,CO]{Novozákonní kantika}

\begin{hora}
\labelText{kantef1}{Ef 1}
\input{generovane/svatecnizaltar/kantikum_ef1.tex}
\columnbreak % ZLOM3
\labelText{kantfp2}{Flp 2}
\input{generovane/svatecnizaltar/kantikum_fp2.tex}
\labelText{kantkol1}{Kol 1}
\input{generovane/svatecnizaltar/kantikum_kol1.tex}
\labelText{kant1tim3}{1 Tim 3}
\begin{psalmus}
\nadpisZalmu{srov. 1 Tim 3, 16}

\rubr{O.} Chvalte Pána, všechny národy.

On přišel v lidské \underline{při}ro\-\underline{ze}nos\-ti,~*
byl o\-spra\-\underline{ve}dl\-něn \underline{Du}chem.

\rubr{O.} Chvalte Pána, všechny národy.

Uká\underline{zal} se \underline{an}\-dě\-lům,~*
byl \underline{hlá}\-sán \underline{po}\-ha\-nům.

\rubr{O.} Chvalte Pána, všechny národy.

Došel \underline{ví}\-ry \underline{ve} světě,~*
\textit{\underline{byl}\zalmVersUpozorneni{2} vzat \underline{do} slávy.}

\rubr{O.} Chvalte Pána, všechny národy.
\end{psalmus}

\labelText{kant1petr2}{1 Petr 2}
\input{generovane/svatecnizaltar/kantikum_1petr2.tex}
\labelText{kantzj4}{Zj 4}
\input{generovane/svatecnizaltar/kantikum_zj4.tex}
\labelText{kantzj11}{Zj 11}
\input{generovane/svatecnizaltar/kantikum_zj11.tex}
\labelText{kantzj15}{Zj 15}
\input{generovane/svatecnizaltar/kantikum_zj15.tex}
\labelText{kantzj19}{Zj 19}
\begin{psalmus}
\nadpisZalmuBezMezery{kantikum\\ srov. Zj 19, 1-7}

Aleluja.
Vítězství, sláva a moc našemu Bohu, 

\rubrika{O.} aleluja

neboť jeho soudy jsou pravdivé a spravedlivé. 

\rubrika{O.} Aleluja, aleluja.

Aleluja. 
Chvalte našeho Boha, všichni, kdo mu sloužíte 

\rubrika{O.} aleluja

a kdo se ho bojíte, malí i velcí! 

\rubrika{O.} Aleluja, aleluja.

Aleluja. 
Pán, náš Bůh vševládný, se ujal království! 

\rubrika{O.} Aleluja.

Radujme se, jásejme a vzdejme mu čest! 

\rubrika{O.} Aleluja, aleluja.

Aleluja. 
Neboť nadešla Beránkova svatba, 

\rubrika{O.} aleluja

jeho nevěsta se připravila. 

\rubrika{O.} Aleluja, aleluja. 

Sláva Otci i Synu i Duchu svatému.

\rubrika{O.} Aleluja.

Jako byla na počátku, i nyní i vždycky, a na věky věků. Amen.

\rubrika{O.} Aleluja, aleluja. 
\end{psalmus}

\end{hora}

\clearpage
\section{Index svátků}
\fancyhead[CE,CO]{Index svátků}

% generovany soubor
\input{generovane/svatecnizaltar/svatecnizaltar_index.txt.index.tex}

\clearpage
\pagestyle{empty} % bez hlavicky a bez cisla stranky
\tableofcontents

\clearpage
\tirazSvazkuAntifonare

\end{document}
