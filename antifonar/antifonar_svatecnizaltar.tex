\documentclass[a5paper, twoside]{article}
\usepackage[utf8]{inputenc}

% cestina hazi pri prekladu error a tak zdrzuje a blokuje automaticke generovani;
% pri tvorbe finalniho vystupu je treba ji odkomentovat a preklad spoustet
% rucne, ne pres rake.
% \usepackage{czech} 

\usepackage[left=2cm, right=1.5cm, top=2cm, bottom=1cm]{geometry} % okraje stranky
\usepackage[colorlinks=true, citecolor=black, linkcolor=black, urlcolor=black]{hyperref} % hypertextove odkazy
\usepackage{datetime} % formaty data
\usepackage{multicol} % sazba ve sloupcich
\usepackage{lettrine} % inicialky
\usepackage{fancyhdr} % hlavicky stranek
\usepackage{color} % barevny text

\usepackage[T1]{fontenc}
% font Palatino:
% \usepackage[sc]{mathpazo}
% \linespread{1.05}        

% font Bookman:
\usepackage{bookman}


% Definice prikazu uzivanych ve vsech svazcich antifonare

% Predpokladane balicky: 
% datetime, hyperref


%
% Titulni strana a tiraz svazku antifonare.
% prikazy vyuzivaji prikazy, ktere musi byt
% definovany v dokumentu:
% \soubornyNazev, \nazev, \cisloSvazku, \autor, \textDoTiraze
%

\newcommand{\titulniStrankaSvazkuAntifonare}{
  \pagestyle{empty} % bez hlavicky a bez cisla stranky
  \begin{titlepage}
    \begin{center}
      \soubornyNazev
      
      Svazeček \cisloSvazku.
      
      \vspace*{7cm}
      
      {\Huge \textbf{\nazev}}
          
      \vspace{1cm}
      
      {\large \autor}
      
      \vfill
      In adiutorium
      
      \onlyyeardate \today
    \end{center}
  \end{titlepage}
}

\newcommand{\tirazProjekt}{
  projekt In adiutorium - hudba k české liturgii hodin\\  
    \url{http://inadiutorium.xf.cz}
}

\newcommand{\lilypondbookVersion}{
  \input{|"lilypond-book --version"}
}

\newcommand{\tirazSazbaProgramem}{
  Vysázeno programy \LaTeX  a LilyPond % \lilypondbookVersion
}

\newcommand{\tirazSvazkuAntifonare}{
  \pagestyle{empty} % bez hlavicky a bez cisla stranky
  \setlength{\parskip}{0.6cm}

  \begin{center}
    \soubornyNazev, svazeček \cisloSvazku.
    
    {\Large \textbf{\nazev}}

    {\large \autor}

    \tirazProjekt
    
    \tirazSazbaProgramem
  \end{center}

  \setlength{\parindent}{0cm} % prvni radek odstavce neodsazovat
  \textDoTiraze

  \begin{center}
    \dmyyyydate \today
  \end{center}
}

%
% Vyrobi prazdnou stranku
%

\newcommand{\prazdnaStranka}{
  \newpage \mbox{}
  \newpage
}

%
% Nasledujici prikazy vyrabeji hlavicku a paticku pro
% male materialy obsahem a razem podobne materialum
% z LilyPondu
%

\newcommand{\malyTitulek}{
  \begin{center}
    {\large \textbf{\nazev}}
  \end{center}
  \begin{flushright}
    \autor
  \end{flushright}
}

\newcommand{\malaTiraz}{
  datum: \dmyyyydate \today
  
  licence: Creative Commons Attribution-ShareAlike 3.0 Unported
  
  projekt: In adiutorium - noty k liturgii hodin (http://inadiutorium.xf.cz)
  
  sazba programy \LaTeX a LilyPond \lilypondbookVersion
}

% cesky format data:
\renewcommand{\dateseparator}{.}

% format data, kde je jen rok
\newdateformat{onlyyeardate}{%
\THEYEAR}


%
% Nasledujici prikazy slouzi strukturovani uvnitr textu antifonare
%

% tyden
\newcommand{\nadpisTyden}[2]{
  \begin{center}
  {\LARGE \textsc{#1}}
  \end{center}
  \fancyhead[CE]{#2}
  \addcontentsline{toc}{section}{#2}
}

% den
\newcommand{\nadpisDen}[1]{
  \begin{center}
  {\LARGE #1}
  \end{center}
  \fancyhead[CO]{#1}
  \addcontentsline{toc}{subsection}{#1}
}

% hodinka
\newcommand{\nadpisHodinka}[1]{
  \begin{center}
  \textbf{#1}
  \end{center}
}

\newcommand{\modlitbaSeCtenim}{
  \nadpisHodinka{Modlitba se čtením}
}

\newcommand{\ranniChvaly}{
  \nadpisHodinka{Ranní chvály}
}

\newcommand{\modlitbaUprostredDne}{
  \nadpisHodinka{Modlitba uprostřed dne}
}

\newcommand{\nespory}{
  \nadpisHodinka{Nešpory}
}

\newcommand{\nesporyI}{
  \nadpisHodinka{První nešpory}
}

\newcommand{\nesporyII}{
  \nadpisHodinka{Druhé nešpory}
}

\newcommand{\kompletar}{
  \nadpisHodinka{Kompletář}
}

% nadpis hodinky uzivany v indexu svatku svatecniho zaltare
\newcommand{\nadpisHodinkaVIndexu}[1]{
  \noindent#1:
  \newline\indent
}

% Varianty tech samych prikazu pro index svatku:
\newcommand{\idxModlitbaSeCtenim}{
  \nadpisHodinkaVIndexu{modlitba se čtením}
}

\newcommand{\idxRanniChvaly}{
  \nadpisHodinkaVIndexu{ranní chvály}
}

\newcommand{\idxModlitbaUprostredDne}{
  \nadpisHodinkaVIndexu{modlitba uprostřed dne}
}

\newcommand{\idxNespory}{
  \nadpisHodinkaVIndexu{nešpory}
}

\newcommand{\idxNesporyI}{
  \nadpisHodinkaVIndexu{první nešpory}
}

\newcommand{\idxNesporyII}{
  \nadpisHodinkaVIndexu{druhé nešpory}
}

\newcommand{\idxKompletar}{
  \nadpisHodinkaVIndexu{kompletář}
}


% Cilem je, aby tento nadpis nad sebou nemel tolik mista
% a tak umoznoval "prilepeni" predchazejici rubriky.
\newcommand{\nadpisZalmuBezMezery}[1]{

  % Nepouziva prostredi center, protoze to si svevolne pridava
  % dodatecny vertikalni prostor
  \begingroup\centering 
    \textbf{#1}
  \endgroup
  \vspace{3mm}
  \nopagebreak
}

% obvykle napr. "zalm 94"
\newcommand{\nadpisZalmu}[1]{
  \vspace{2mm}
  \begin{center}
    \textbf{#1}
  \end{center}
  \nopagebreak
}


% Napr. "hymnus", "responsorium", "zalmy"
\newcommand{\nadpisTypTextu}[1]{
  \begin{flushleft}
  \noindent
  \rubr{\textsc{#1}}
  \end{flushleft}
  \nopagebreak
}

% rubrikovany text - treba i pro vlozeni do nerubrikovaneho
\newcommand{\rubr}[1]{
  \textcolor{red}{#1}
}

% rubrika - samostatna
\newcommand{\rubrika}[1]{
  \vspace{2mm}
  % \textit{#1}
  \rubr{#1}
  \vspace{2mm}
}

% mezera, kterou se oddeli rubrika od textu, ke kteremu se _nevztahuje_,
% aby byla vic prilepena k tomu, ke kteremu se vztahuje
\newcommand{\rubrikaMezera}{
  \vspace{4mm}
}

% rubrika - predchazejici
\newcommand{\rubrikaPred}[1]{
  \rubrikaMezera
  % \textit{#1}
  \rubr{#1}
  \vspace{1mm}
}

% rubrika - nasledujici
\newcommand{\rubrikaPo}[1]{
  \vspace{1mm}
  % \textit{#1}
  \rubr{#1}
  \rubrikaMezera
}

\newcommand{\znackaStrofaZalmu}{ 
  \rubr{--}
}


\newcommand{\autor}{}
\newcommand{\soubornyNazev}{Antifonář k Denní modlitbě církve}
\newcommand{\cisloSvazku}{5}
\newcommand{\nazev}{Žaltář pro slavnosti a svátky}

\renewcommand{\tirazSazbaProgramem}{
  Vysázeno programem \LaTeX.
}
\newcommand{\textDoTiraze}{
  Texty Denní modlitby církve jsou majetkem České biskupské konference. 
  
  Texty žalmů a kantik jsou převzaté z webu \url{http://ebreviar.cz},
  jehož tvůrcům za jejich práci tímto srdečně děkuji.
  Z textů je vypuštěno to, co se zdá být pro modlitbu v chóru
  nadbytečné nebo rušivé (číslování žalmů i podle Vulgáty,
  nadpisy žalmů a připojené citáty z Písma a Otců)
  a kde jsem shledal odchylky od vydání breviáře z r. 1994, jsou
  podle něj opraveny.
  
  Text žalmů a kantik značkami opatřil a k sazbě připravil Jakub Pavlík.
}

\newenvironment{hora}{
  \begin{multicols}{2}
}{
  \end{multicols}
}

\newcommand{\rubrikaZalmBylvInvitatoriu}{
  \rubrika{Pokud se následující žalm zpíval jako invitatorium,
  nahradí se zde žalmem 95, str. \pageref{zalm24}.}
}

\begin{document}

\pagestyle{empty}

\titulniStrankaSvazkuAntifonare

\prazdnaStranka

% Nastaveni sazby do sloupcu:
\setlength{\columnseprule}{1pt} % cara oddelujici sloupce
\setlength{\columnsep}{20pt} % prostor mezi sloupci

\fancyhead{}
\fancyhead[LE,RO]{\thepage}
\fancyfoot{}

\pagestyle{fancy}

\section{Úvod}

Tento žaltář je určen pro dny, kdy se v breviáři nepoužívají žalmy
z cyklu žaltáře, ale ze společných textů o svatých nebo z vlastních textů 
slavnosti nebo svátku.

V zájmu úspory místa a za cenu jistého zmenšení uživatelského pohodlí
nejsou žalmy a kantika řazeny jako v breviáři a neopakují se,
nýbrž jsou v příslušných oddílech řazeny podle pořadí v katolickém
biblickém kánonu. Na konci knihy je přehledný seznam všech slavností
a svátků, které mají vlastní žalmy, spolu s odkazy na texty.

Doporučuji opatřit výtisky dostatečným množstvím záložek (ideální se
zdají být 3-4).

\nadpisTyden{Kantika z evangelií}{Kantika z evangelií}
\fancyhead[CE,CO]{Kantika z evangelií}

\begin{hora}
\nadpisTypTextu{Zachariášovo kantikum (Benedictus)}
\input{generovane/zaltar/kantikum_benedictus.tex}
\columnbreak
\nadpisTypTextu{Kantikum Panny Marie (Magnificat)}
\input{generovane/zaltar/kantikum_magnificat.tex}
\end{hora}

% \section*{Žalmy invitatoria}
\nadpisTyden{Žalmy invitatoria}{Žalmy invitatoria}
\fancyhead[CE,CO]{Žalmy invitatoria}

\begin{hora}
\input{generovane/zaltar/zalm95.tex}
\input{generovane/zaltar/zalm100.tex}
\input{generovane/zaltar/zalm67.tex}
\input{generovane/zaltar/zalm24.tex}
\label{zalm24}
\end{hora}

\clearpage

\section{Často užívané celky}

\nadpisDen{Neděle 1. týdne - ranní chvály}

\ranniChvaly
\begin{hora}
\input{generovane/zaltar/zalm63.tex}
\input{generovane/zaltar/kantikum_dan3i.tex}
\input{generovane/zaltar/zalm149.tex}
\end{hora}

\clearpage
\fancyhead{} % vynulovat hlavicky
\nadpisTyden{Doplňovací cyklus žalmů pro modlitbu uprostřed dne}{Doplňovací cyklus žalmů}
\fancyhead[CE,CO]{Doplňovací cyklus žalmů}
  
\nadpisHodinka{První oddíl (dopoledne)}
\begin{hora}
\input{generovane/zaltar/zalm120.tex}
\input{generovane/zaltar/zalm121.tex}
\input{generovane/zaltar/zalm122.tex}
\end{hora}

\nadpisHodinka{Druhý oddíl (v poledne)}
\begin{hora}
\input{generovane/zaltar/zalm123.tex}
\input{generovane/zaltar/zalm124.tex}
\input{generovane/zaltar/zalm125.tex}
\end{hora}

\nadpisHodinka{Třetí oddíl (odpoledne)}
\begin{hora}
\input{generovane/zaltar/zalm126.tex}
\input{generovane/zaltar/zalm127.tex}
\input{generovane/zaltar/zalm128.tex}
\end{hora}

\section{Žalmy}

\section{Starozákonní kantika}

\section{Novozákonní kantika}

\begin{hora}
\input{generovane/zaltar/kantikum_ef1.tex}
\input{generovane/zaltar/kantikum_fp2.tex}
\input{generovane/zaltar/kantikum_kol1.tex}
\input{generovane/zaltar/kantikum_zj4.tex}
\input{generovane/zaltar/kantikum_zj11.tex}
\input{generovane/zaltar/kantikum_zj15.tex}
\begin{psalmus}
\nadpisZalmuBezMezery{kantikum\\ srov. Zj 19, 1-7}

Aleluja.
Vítězství, sláva a moc našemu Bohu, 

\rubrika{O.} aleluja

neboť jeho soudy jsou pravdivé a spravedlivé. 

\rubrika{O.} Aleluja, aleluja.

Aleluja. 
Chvalte našeho Boha, všichni, kdo mu sloužíte 

\rubrika{O.} aleluja

a kdo se ho bojíte, malí i velcí! 

\rubrika{O.} Aleluja, aleluja.

Aleluja. 
Pán, náš Bůh vševládný, se ujal království! 

\rubrika{O.} Aleluja.

Radujme se, jásejme a vzdejme mu čest! 

\rubrika{O.} Aleluja, aleluja.

Aleluja. 
Neboť nadešla Beránkova svatba, 

\rubrika{O.} aleluja

jeho nevěsta se připravila. 

\rubrika{O.} Aleluja, aleluja. 

Sláva Otci i Synu i Duchu svatému.

\rubrika{O.} Aleluja.

Jako byla na počátku, i nyní i vždycky, a na věky věků. Amen.

\rubrika{O.} Aleluja, aleluja. 
\end{psalmus}

\end{hora}

\section{Index svátků}

\subsection{Temporál}

\subsection{Společné texty}

\subsection{Sanktorál}

\section{Index všech obsažených textů}

\clearpage
\pagestyle{empty} % bez hlavicky a bez cisla stranky
\tableofcontents

\clearpage
\tirazSvazkuAntifonare

\end{document}