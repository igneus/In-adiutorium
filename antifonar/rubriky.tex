% Casto opakovane standardni odkazy

\newcommand{\rubrikaZalmBylvInvitatoriu}{
  \rubrika{Pokud se následující žalm zpíval jako invitatorium,
  nahradí se zde žalmem 95, str.~\pageref{zalm24}.}
}

\newcommand{\rubrikaBenedictus}{
  \begin{flushleft}
  \noindent
  \rubrika{\textsc{Zachariášovo kantikum,} str.~\pageref{benedictus}.}
  \end{flushleft}
}

\newcommand{\rubrikaMagnificat}{
  \begin{flushleft}
  \noindent
  \rubrika{\textsc{Kantikum Panny Marie,} str.~\pageref{magnificat}.}
  \end{flushleft}
}

\newcommand{\rubrikaInvitatorium}{
  \rubrika{Jsou-li ranní chvály první modlitbou dne, předchází jim
  uvedení do první modlitby dne, str.~\pageref{invitatorium}.}
}

\newcommand{\rubrikaInvitatoriumMc}{
  \rubrika{Je-li modlitba se čtením první modlitbou dne, předchází jí
  uvedení do první modlitby dne, str.~\pageref{invitatorium}.}
}

\newcommand{\rubrikaDoplnovaciCyklus}{
  \rubrika{Doplňovací cyklus žalmů je na str.~\pageref{doplnovacicyklus}}
}

\newcommand{\rubrikaNebo}{
  \rubrika{Nebo:}
}

% bezne nadpisy typu textu

\newcommand{\typZalmy}{
  \nadpisTypTextu{Žalmy}
}
\newcommand{\typResponsorium}{
  \nadpisTypTextu{Responsorium}
}
\newcommand{\typResponsoriumMcI}{
  \nadpisTypTextu{Responsorium po prvním čtení}
}
\newcommand{\typResponsoriumMcII}{
  \nadpisTypTextu{Responsorium po druhém čtení}
}
\newcommand{\typVers}{
  \nadpisTypTextu{Verš}
}
\newcommand{\typMagnificat}{
  \nadpisTypTextu{Kantikum Panny Marie}
}
\newcommand{\typBenedictus}{
  \nadpisTypTextu{Zachariášovo kantikum}
}
\newcommand{\typTeDeum}{
  \nadpisTypTextu{Chvalozpěv Bože, tebe chválíme}
}
