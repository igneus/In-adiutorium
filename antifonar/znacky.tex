% znacky.tex
%
% potrebuje baliky: zref-savepos, changepage, etoolbox
%
% Umoznuje ve dvousloupcovem usporadani stranky umistit na okraj
% (pravy nebo levy, podle sloupce) varovnou znacku.

% Nastaveni dulezita pro spravne fungovani:
% -----------------------------------
\strictpagecheck % pro makra tvorici vystrazne znacky

\newcounter{zalmVersUpozorneni}
\setcounter{zalmVersUpozorneni}{1}
% ------------------------------------

% konstanty uzivane v nasledujicich makrech
\def\pulstranky{13773045}%sp; vypocteno; pul stranky A5
\def\vzdalenostZnackyOdTextu{0.2}%cm

% argument: label ulozene pozice. Vrati hodnotu pro \hskip na levy okr. stranky
\def\kLevemuOkraji#1{%
  -\zposx{#1}%
}

% Macro written for me by Yiannis Lazarides
% http://tex.stackexchange.com/questions/50216/margin-notes-in-a-multicolumn-environment
% I modified it slightly for my needs.
%
% uses package zref-savepos
%
% Arguments: 
% #1 unique label identifier
% #2 text
%
% In the two-columned setup it puts given text besides the column
% where the macro is called
%
% Numbers which depend on the page's geometry!!!!!!!:
% 1.5cm & 1.1cm: margins of the page (I don't want to have them hardcoded,
% but I haven't found any way to determine them at runtime.)
\def\putmarginpar#1#2{%
  \zsavepos{#1}%
  \checkoddpage
  \ifnum\pulstranky>\number\zposx{#1}%
    % tady je dimen100 levy okraj stranek
    \ifoddpage% liche - siroky
      \dimen100=1.5cm%
    \else%
      \dimen100=1.1cm%
    \fi%
    \hbox to 0pt{\hskip\dimexpr\kLevemuOkraji{#1}sp + \dimen100 - \vzdalenostZnackyOdTextu cm \relax#2}%
  \else
    % tady je dimen100 pravy okraj stranek
    \ifoddpage% liche - uzky
      \dimen100=1.1cm%
    \else%
      \dimen100=1.5cm%
    \fi%
     \hbox to 0pt{\hskip\dimexpr\kLevemuOkraji{#1}sp +\pulstranky sp * 2 - \dimen100 + \vzdalenostZnackyOdTextu cm \relax#2}%
  \fi%
}

%
% Znacka: varovny vykricnik na okraji stranky vedle aktualni radky
% (Prikaz predpoklada dvousloupcove usporadani, v jednosloupcovem bude
% pracovat nelogicky)
%
% Jediny argument je cislo - zavaznost anomalie, na kterou se
% upozornuje:
% 3 =  nema prvni akcent -> problem maji nejen dvouakcentove napevy,
%      ale i vsechny jednoakcentove s pripravou > 1
% 2 =  prvni akcent na prvni slabice -> dvouakcentove napevy bez pripravy
%      jdou, ale je to tezke; dvouakcentove s pripravou maji problem;
%      jednoakcentove s delsi pripravou take

\def\txtZnackaPrikaz#1{\textup{\footnotesize{\textbf{#1}}}}
\newcommand{\zalmVersUpozorneni}[1]{%
  \stepcounter{zalmVersUpozorneni}%
  \ifnumequal{#1}{3}{%
    \putmarginpar{znac:\arabic{zalmVersUpozorneni}}{\txtZnackaPrikaz{!!}}%
  }{% else
    \putmarginpar{znac:\arabic{zalmVersUpozorneni}}{\txtZnackaPrikaz{!}}%
  }%
}
