\documentclass[a5paper]{article}
\usepackage[utf8]{inputenc}
% \usepackage{czech}
\usepackage[margin=1.5cm]{geometry} % okraje stranky
\usepackage[colorlinks=true]{hyperref} % hypertextove odkazy
\usepackage{datetime} % formaty data

\newcommand{\autor}{Jakub Pavlík}
\newcommand{\soubornyNazev}{Antifonář k Denní modlitbě církve}
\newcommand{\cisloSvazku}{2}
\newcommand{\nazev}{Žaltář}

\newcommand{\textDoTiraze}{
  Texty Denní modlitby církve jsou majetkem České biskupské konference. 
  
  Texty žalmů a kantik jsou převzaté z webu ebreviar.cz,
  jehož tvůrcům za jejich práci tímto srdečně děkuji.
  Z textů je vypuštěno to, co se zdá být pro modlitbu v chóru
  nadbytečné nebo rušivé (číslování žalmů i podle Vulgáty,
  nadpisy žalmů a připojené citáty z Písma a Otců)
  a kde jsem shledal odchylky od vydání breviáře z r. 1994, jsou
  podle něj opraveny.
}

\newcommand{\preLilyPondExample}{ \begin{flushleft} }
\newcommand{\postLilyPondExample}{ \end{flushleft} }

\begin{document}

% Definice prikazu uzivanych ve vsech svazcich antifonare

% Predpokladane balicky: 
% datetime, hyperref


%
% Titulni strana a tiraz svazku antifonare.
% prikazy vyuzivaji prikazy, ktere musi byt
% definovany v dokumentu:
% \soubornyNazev, \nazev, \cisloSvazku, \autor, \textDoTiraze
%

\newcommand{\titulniStrankaSvazkuAntifonare}{
  \pagestyle{empty} % bez hlavicky a bez cisla stranky
  \begin{titlepage}
    \begin{center}
      \soubornyNazev
      
      Svazeček \cisloSvazku.
      
      \vspace*{7cm}
      
      {\Huge \textbf{\nazev}}
          
      \vspace{1cm}
      
      {\large \autor}
      
      \vfill
      In adiutorium
      
      \onlyyeardate \today
    \end{center}
  \end{titlepage}
}

\newcommand{\tirazProjekt}{
  projekt In adiutorium - hudba k české liturgii hodin\\  
    \url{http://inadiutorium.xf.cz}
}

\newcommand{\lilypondbookVersion}{
  \input{|"lilypond-book --version"}
}

\newcommand{\tirazSazbaProgramem}{
  Vysázeno programy \LaTeX  a LilyPond % \lilypondbookVersion
}

\newcommand{\tirazSvazkuAntifonare}{
  \pagestyle{empty} % bez hlavicky a bez cisla stranky
  \setlength{\parskip}{0.6cm}

  \begin{center}
    \soubornyNazev, svazeček \cisloSvazku.
    
    {\Large \textbf{\nazev}}

    {\large \autor}

    \tirazProjekt
    
    \tirazSazbaProgramem
  \end{center}

  \setlength{\parindent}{0cm} % prvni radek odstavce neodsazovat
  \textDoTiraze

  \begin{center}
    \dmyyyydate \today
  \end{center}
}

%
% Vyrobi prazdnou stranku
%

\newcommand{\prazdnaStranka}{
  \newpage \mbox{}
  \newpage
}

%
% Nasledujici prikazy vyrabeji hlavicku a paticku pro
% male materialy obsahem a razem podobne materialum
% z LilyPondu
%

\newcommand{\malyTitulek}{
  \begin{center}
    {\large \textbf{\nazev}}
  \end{center}
  \begin{flushright}
    \autor
  \end{flushright}
}

\newcommand{\malaTiraz}{
  datum: \dmyyyydate \today
  
  licence: Creative Commons Attribution-ShareAlike 3.0 Unported
  
  projekt: In adiutorium - noty k liturgii hodin (http://inadiutorium.xf.cz)
  
  sazba programy \LaTeX a LilyPond \lilypondbookVersion
}

% cesky format data:
\renewcommand{\dateseparator}{.}

% format data, kde je jen rok
\newdateformat{onlyyeardate}{%
\THEYEAR}


%
% Nasledujici prikazy slouzi strukturovani uvnitr textu antifonare
%

% tyden
\newcommand{\nadpisTyden}[2]{
  \begin{center}
  {\LARGE \textsc{#1}}
  \end{center}
  \fancyhead[CE]{#2}
  \addcontentsline{toc}{section}{#2}
}

% den
\newcommand{\nadpisDen}[1]{
  \begin{center}
  {\LARGE #1}
  \end{center}
  \fancyhead[CO]{#1}
  \addcontentsline{toc}{subsection}{#1}
}

% hodinka
\newcommand{\nadpisHodinka}[1]{
  \begin{center}
  \textbf{#1}
  \end{center}
}

\newcommand{\modlitbaSeCtenim}{
  \nadpisHodinka{Modlitba se čtením}
}

\newcommand{\ranniChvaly}{
  \nadpisHodinka{Ranní chvály}
}

\newcommand{\modlitbaUprostredDne}{
  \nadpisHodinka{Modlitba uprostřed dne}
}

\newcommand{\nespory}{
  \nadpisHodinka{Nešpory}
}

\newcommand{\nesporyI}{
  \nadpisHodinka{První nešpory}
}

\newcommand{\nesporyII}{
  \nadpisHodinka{Druhé nešpory}
}

\newcommand{\kompletar}{
  \nadpisHodinka{Kompletář}
}

% nadpis hodinky uzivany v indexu svatku svatecniho zaltare
\newcommand{\nadpisHodinkaVIndexu}[1]{
  \noindent#1:
  \newline\indent
}

% Varianty tech samych prikazu pro index svatku:
\newcommand{\idxModlitbaSeCtenim}{
  \nadpisHodinkaVIndexu{modlitba se čtením}
}

\newcommand{\idxRanniChvaly}{
  \nadpisHodinkaVIndexu{ranní chvály}
}

\newcommand{\idxModlitbaUprostredDne}{
  \nadpisHodinkaVIndexu{modlitba uprostřed dne}
}

\newcommand{\idxNespory}{
  \nadpisHodinkaVIndexu{nešpory}
}

\newcommand{\idxNesporyI}{
  \nadpisHodinkaVIndexu{první nešpory}
}

\newcommand{\idxNesporyII}{
  \nadpisHodinkaVIndexu{druhé nešpory}
}

\newcommand{\idxKompletar}{
  \nadpisHodinkaVIndexu{kompletář}
}


% Cilem je, aby tento nadpis nad sebou nemel tolik mista
% a tak umoznoval "prilepeni" predchazejici rubriky.
\newcommand{\nadpisZalmuBezMezery}[1]{

  % Nepouziva prostredi center, protoze to si svevolne pridava
  % dodatecny vertikalni prostor
  \begingroup\centering 
    \textbf{#1}
  \endgroup
  \vspace{3mm}
  \nopagebreak
}

% obvykle napr. "zalm 94"
\newcommand{\nadpisZalmu}[1]{
  \vspace{2mm}
  \begin{center}
    \textbf{#1}
  \end{center}
  \nopagebreak
}


% Napr. "hymnus", "responsorium", "zalmy"
\newcommand{\nadpisTypTextu}[1]{
  \begin{flushleft}
  \noindent
  \rubr{\textsc{#1}}
  \end{flushleft}
  \nopagebreak
}

% rubrikovany text - treba i pro vlozeni do nerubrikovaneho
\newcommand{\rubr}[1]{
  \textcolor{red}{#1}
}

% rubrika - samostatna
\newcommand{\rubrika}[1]{
  \vspace{2mm}
  % \textit{#1}
  \rubr{#1}
  \vspace{2mm}
}

% mezera, kterou se oddeli rubrika od textu, ke kteremu se _nevztahuje_,
% aby byla vic prilepena k tomu, ke kteremu se vztahuje
\newcommand{\rubrikaMezera}{
  \vspace{4mm}
}

% rubrika - predchazejici
\newcommand{\rubrikaPred}[1]{
  \rubrikaMezera
  % \textit{#1}
  \rubr{#1}
  \vspace{1mm}
}

% rubrika - nasledujici
\newcommand{\rubrikaPo}[1]{
  \vspace{1mm}
  % \textit{#1}
  \rubr{#1}
  \rubrikaMezera
}

\newcommand{\znackaStrofaZalmu}{ 
  \rubr{--}
}


\titulniStrankaSvazkuAntifonare

Tento žaltář obsahuje texty žalmů a kantik ze žaltáře
Denní modlitby církve s označkovanými přízvuky důležitými
pro zpěv.
Je na jednu stranu přípravou pro žaltář s notovanými antifonami,
zároveň však jeho univerzálnějším doplňkem - lze jej používat
s různými sadami antifon.

% \input{generovane/.tex}

\section*{Kantika z evangelií}

\nadpisTypTextu{Zachariášovo kantikum (Benedictus)}
\input{generovane/kantikum_benedictus.tex}
\nadpisTypTextu{Kantikum Panny Marie (Magnificat)}
\input{generovane/kantikum_magnificat.tex}

\section*{Žalmy invitatoria}

\input{generovane/zalm95.tex}
\input{generovane/zalm100.tex}
\input{generovane/zalm67.tex}
\input{generovane/zalm24.tex}

\section*{Čtyřtýdenní cyklus žaltáře}

\nadpisTyden{První týden}

\nadpisDen{Neděle}

\nesporyI

\input{generovane/zalm141.tex}
\input{generovane/zalm142.tex}
\input{generovane/kantikum_fp2.tex}

\ranniChvaly

\input{generovane/zalm63.tex}
\input{generovane/kantikum_dan3i.tex}
\input{generovane/zalm149.tex}

\modlitbaUprostredDne

\input{generovane/zalm118i.tex}
\input{generovane/zalm118ii.tex}
\input{generovane/zalm118iii.tex}

\nesporyII

\input{generovane/zalm110.tex}
\input{generovane/zalm114.tex}
\begin{psalmus}
\nadpisZalmuBezMezery{kantikum\\ srov. Zj 19, 1-7}

Aleluja.
Vítězství, sláva a moc našemu Bohu, 

\rubrika{O.} aleluja

neboť jeho soudy jsou pravdivé a spravedlivé. 

\rubrika{O.} Aleluja, aleluja.

Aleluja. 
Chvalte našeho Boha, všichni, kdo mu sloužíte 

\rubrika{O.} aleluja

a kdo se ho bojíte, malí i velcí! 

\rubrika{O.} Aleluja, aleluja.

Aleluja. 
Pán, náš Bůh vševládný, se ujal království! 

\rubrika{O.} Aleluja.

Radujme se, jásejme a vzdejme mu čest! 

\rubrika{O.} Aleluja, aleluja.

Aleluja. 
Neboť nadešla Beránkova svatba, 

\rubrika{O.} aleluja

jeho nevěsta se připravila. 

\rubrika{O.} Aleluja, aleluja. 

Sláva Otci i Synu i Duchu svatému.

\rubrika{O.} Aleluja.

Jako byla na počátku, i nyní i vždycky, a na věky věků. Amen.

\rubrika{O.} Aleluja, aleluja. 
\end{psalmus}

  
\clearpage
\tirazSvazkuAntifonare

\end{document}