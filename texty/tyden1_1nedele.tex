% nedele 1. tydne

\subsection{Neděle}

% -------------------------------------------

\subsubsection{První nešpory}

\nadpisZalm{Žalm 141 (140), 1-9}

Hospodine, volám k tobě, \underline{po}spěš mi \underline{na} pomoc,~*
slyš můj hlas, když \underline{k to}bě \underline{vo}lám.

Má modlitba ať je před tebou \underline{ja}ko \underline{ka}didlo,~*
mé zvednuté dlaně jako \underline{ve}černí \underline{o}běť.

Postav, Hospodine, \underline{stráž} k mým \underline{ús}tům,~*
hlídku \underline{k brá}ně \underline{mých} rtů.

Neskloň mé \underline{srd}ce \underline{ke} zlému,~*
abych se nedo\underline{pouš}těl \underline{zlo}činů,

abych s pacha\underline{te}li \underline{bez}práví~*
nejídal jejich \underline{vy}braná \underline{jíd}la.

Když mě udeří spravedlivý, \underline{je} to \underline{las}kavost,~*
když mě pohaní, je to \underline{pro} hlavu \underline{o}lej.

Neodmítne \underline{ho} má \underline{hla}va,~*
avšak pod tíží jejich zloby stále se \underline{bu}du \underline{mod}lit.

U skály byla propuštěna \underline{je}jich \underline{kní}žata,~*
slyšela, jak laskavá \underline{by}la má \underline{slo}va.

Jako když se oře a \underline{vlá}čí \underline{pů}da,~*
jejich kosti byly rozmetány \underline{u} jícnu \underline{pod}světí.

K tobě, Hospodine, Pane, se obrace\underline{jí} mé \underline{o}či,~*
k tobě se utíkám, nevy\underline{dá}vej mě \underline{smr}ti!

Zachraň mě před léčkou, kterou mi \underline{na}stra\underline{ži}li,~*
před tenaty \underline{pa}cha\underline{te}lů křivd.

\nadpisZalm{Žalm 142 (141)}

Hlasitě k Hospo\underline{di}nu \underline{vo}lám,~*
hlasitě Hospodina \underline{za}pří\underline{sa}hám.

Vylévám \underline{před} ním svou \underline{sta}rost,~*
svou tíseň mu \underline{vy}pra\underline{vu}ji.

Když můj duch \underline{ve} mně \underline{chřad}ne,~*
ty \underline{znáš} mou \underline{ces}tu.

Na stezce, \underline{po} níž \underline{krá}čím,~*
mi nastra\underline{ži}li \underline{léč}ku.

Hledím na\underline{pra}vo a \underline{vi}dím,~*
že nikdo nemá \underline{na} mě \underline{o}hled.

Nemám \underline{se} kam \underline{u}téci,~*
nikdo se nestará \underline{o} můj \underline{ži}vot.

Volám k tobě, Hospo\underline{di}ne,~+
říkám: Tys mé \underline{ú}to\underline{čiš}tě,~*
můj úděl \underline{v ze}mi \underline{ži}vých.

Všimni si \underline{mé}ho \underline{nář}ku,~*
vždyť jsem tak \underline{zbě}do\underline{va}ný.

Vysvoboď mě od těch, kdo mě \underline{pro}ná\underline{sle}dují,~*
jsou sil\underline{něj}ší \underline{než} já.

Vyveď mě \underline{ze} ža\underline{lá}ře,~*
abych \underline{chvá}lil tvé \underline{jmé}no.

Obklopí mě \underline{spra}ved\underline{li}ví,~*
\underline{až} se mě \underline{u}jmeš.

\nadpisZalm{Kantikum Flp 2, 6-11}

Kristus Ježíš, ačkoli má božskou \underline{při}ro\underline{ze}nost,~*
nic nelpěl na tom, že je \underline{rov}ný \underline{Bo}hu,

ale sám sebe se \underline{zře}kl,~+
vzal na sebe přirozenost \underline{slu}žeb\underline{ní}ka~*
a stal se \underline{jed}ním \underline{z li}dí.

Byl jako každý jiný \underline{člo}věk,~+
ponížil se a byl posluš\underline{ný} až \underline{k smr}ti,~*
a to \underline{k smr}ti \underline{na} kříži.

Proto ho \underline{ta}ké Bůh \underline{po}výšil~*
a dal mu Jméno nad každé \underline{ji}né \underline{jmé}no,

takže při Ježíšově jménu musí pokleknout \underline{kaž}dé \underline{ko}leno~*
na nebi, na ze\underline{mi} i \underline{v pod}světí

a každý jazyk musí k slávě Boha \underline{Ot}ce \underline{vy}znat:~*
Ježíš \underline{Kris}tus \underline{je} Pán.

% -------------------------------------------

\subsubsection{Ranní chvály}

\nadpisZalm{Žalm 63 (62), 2-9}

Bože, \underline{ty} jsi \underline{můj} Bůh,~*
\underline{snaž}ně tě \underline{hle}dám,

má duše po tobě žízní, prahne po to\underline{bě} mé \underline{tě}lo~*
jak vyprahlá, žíznivá, \underline{bez}vodá \underline{ze}mě.

Tak toužím tě spatřit \underline{ve} sva\underline{ty}ni,~*
abych viděl tvou \underline{moc} a \underline{slá}vu.

Vždyť tvá milost je \underline{lep}ší než \underline{ži}vot,~*
mé rty tě \underline{bu}dou \underline{chvá}lit.

Tak tě budu velebit \underline{ve} svém \underline{ži}votě,~*
v tvém jménu povznesu své \underline{dla}ně \underline{k mod}litbě.

Má duše se bude sytit jak \underline{tu}kem a \underline{mor}kem,~*
plesajícími rty zajá\underline{sa}jí \underline{ús}ta,

kdykoli na tebe vzpomenu \underline{na} svém \underline{lůž}ku,~*
v nočních hodinách budu \underline{na} tebe \underline{my}slet.

Neboť stal ses mým \underline{po}moc\underline{ní}kem~*
a ve stínu tvých \underline{kří}del \underline{já}sám.

Má \underline{du}še lne \underline{k to}bě,~*
tvá pravi\underline{ce} mě \underline{pod}pírá.

\nadpisZalm{Kantikum Dan 3, 57-88}

\newcommand{\velebteA}{\underline{ve}lebte \underline{Pá}na,~* }
\newcommand{\velebteB}{\underline{ve}lebte \underline{Pá}na. }

Všechna díla Páně, \velebteA
chvalte a oslavuj\underline{te} ho \underline{na}věky.

Nebesa, \velebteA
andělé Páně, \velebteB

Všechny vody nad nebem, \velebteA
všechny mocnosti Páně, \velebteB

Slunce a měsíci, \velebteA
nebeské hvězdy, \velebteB

Všechny deště a roso, \velebteA
všechny větry, \velebteB

Ohni a žáre, \velebteA
studeno a teplo, \velebteB

Roso a jíní, \velebteA
zimo a chlade, \velebteB

Ledy a sněhy, \velebteA
noci a dni, \velebteB

Světlo a temno, \velebteA
blesky a mraky, \velebteB

Země, \underline{ve}leb \underline{Pá}na,~*
chval a osla\underline{vuj} ho \underline{na}věky.

Hory a vrchy, \velebteA
vše, co na zemi roste, \velebteB

Prameny, \velebteA
moře a řeky, \velebteB

Velké a malé ryby, které plují ve vodě, \velebteA
všichni nebeští ptáci, \velebteB

Všechna zvířata divoká i krotká, \velebteA
lidé, \velebteB

Izraeli, \underline{ve}leb \underline{Pá}na,~*
chval a osla\underline{vuj} ho \underline{na}věky.

Kněží Páně, \velebteA
služebníci Páně, \velebteB

Duchové a duše spravedlivých, \velebteA
svatí a pokorní srdcem, \velebteB

Ananiáši, Azariáši, Misaeli, \velebteA
chvalte a oslavuj\underline{te} ho \underline{na}věky.

Velebme Otce i Syna i \underline{Du}cha \underline{sva}tého,~*
chvalme a oslavuj\underline{me} ho \underline{na}věky.

\rubrika{Na konci tohoto kantika se nepřipojuje zakončení Sláva Otci.}

\nadpisZalm{Žalm 149}

Zpívejte Hospodinu \underline{pí}seň \underline{no}vou,~*
jeho chvála ať zaznívá \underline{ve} sboru \underline{sva}tých.

Ať se raduje Izrael ze \underline{své}ho \underline{tvůr}ce,~*
synové Siónu ať jásají \underline{nad} svým \underline{krá}lem.

Ať chválí jeho \underline{jmé}no \underline{tan}cem,~*
ať mu hrají na buben a \underline{na} ci\underline{te}ru,

neboť Hospodin milu\underline{je} svůj \underline{ná}rod~*
a pokorné \underline{zdo}bí \underline{ví}tězstvím.

Ať svatí jásají \underline{chva}lo\underline{zpě}vem,~*
ať se veselí \underline{na} svých \underline{lo}žích.

Boží chválu ať \underline{ma}jí \underline{v hrd}lech~*
a dvojseč\underline{ný} meč \underline{v ru}kou,

aby vykonali pomstu \underline{na} po\underline{ha}nech,~*
tresty \underline{na} ná\underline{ro}dech,

aby spoutali jejich \underline{krá}le \underline{ře}tězy~*
a železnými okovy \underline{je}jich \underline{vel}može,

aby na nich vykonali \underline{ur}če\underline{ný} soud.~*
Všem jeho svatým \underline{bu}de to \underline{ke} cti.

% -------------------------------------------

\subsubsection{Modlitba uprostřed dne}

% -------------------------------------------

\subsubsection{Druhé nešpory}