\documentclass[a5paper, twoside]{article}
\usepackage[utf8]{inputenc}

\usepackage[czech]{babel} 

\usepackage[left=2cm, right=1.5cm, top=2cm, bottom=1cm]{geometry} % okraje stranky
\usepackage[colorlinks=true, citecolor=black, linkcolor=black, urlcolor=black]{hyperref} % hypertextove odkazy
\usepackage{datetime} % formaty data

% \usepackage[T1]{fontenc}
% \usepackage{bookman}


% Definice prikazu uzivanych ve vsech svazcich antifonare

% Predpokladane balicky: 
% datetime, hyperref

%
% spolecna nastaveni
%

\setlength{\parindent}{0.4cm}

%
% Titulni strana a tiraz svazku antifonare.
% prikazy vyuzivaji prikazy, ktere musi byt
% definovany v dokumentu:
% \soubornyNazev, \edicniRada, \nazev, \cisloSvazku, \autor, \textDoTiraze
%

\newcommand{\titulniStrankaSvazkuAntifonare}{
  \pagestyle{empty} % bez hlavicky a bez cisla stranky
  \begin{titlepage}
    \begin{center}
      \soubornyNazev

      \vspace{0.5cm}

      \edicniRada
      
      svazek \cisloSvazku
      
      \vspace*{6cm}
      
      {\Huge \textbf{\nazev}}
          
      \vspace{1cm}
      
      {\large \autor}
      
      \vfill
      In adiutorium
      
      \onlyyeardate \today
    \end{center}
  \end{titlepage}
}

\newcommand{\patitulSvazkuAntifonare}{
  \begin{center}
    \soubornyNazev

    \nazev
  \end{center}

  \vfill
}

\newcommand{\frontispisSvazkuAntifonare}{
  \mbox{}
}

\newcommand{\tirazProjekt}{
  projekt In adiutorium - hudba k české liturgii hodin\\  
    \url{http://inadiutorium.xf.cz}
}

\newcommand{\lilypondbookVersion}{
  \input{|"lilypond-book --version"}
}

\newcommand{\tirazSazbaProgramem}{
  Sazba programy \LaTeX\mbox{} a LilyPond % \lilypondbookVersion
}

\newcommand{\tirazSvazkuAntifonare}{
  \pagestyle{empty} % bez hlavicky a bez cisla stranky
  \setlength{\parskip}{0.6cm}

  \begin{center}
    \soubornyNazev\\
    \edicniRada, svazek \cisloSvazku
    
    {\Large \textbf{\nazev}}

    {\large \autor}

    \tirazProjekt
    
    \tirazSazbaProgramem
  \end{center}

  \setlength{\parindent}{0cm} % prvni radek odstavce neodsazovat
  \textDoTiraze

  \begin{center}
    \dmyyyydate \today
  \end{center}
}

%
% Standartni licencni poznamka pouzitelna pro vetsinu svazku
% Antifonare k Denni modlitbe cirkve
%

\newcommand{\licencniPoznamka}{%
  Používání \emph{nápěvů} je vázáno licencí 
  \href{http://creativecommons.org/licenses/by-sa/3.0/deed.cs}{Creative Commons 
    At\-tri\-bu\-tion-\-Sha\-re-A\-li\-ke 3.0 Unported},
  která dává komukoli právo je dále šířit, upravovat a využívat
  ve svých vlastních dílech, za předpokladu, že uvede informaci
  o autorovi původního díla a při dalším šíření zachová původní licenční 
  podmínky.
  (Výjimku představují nápěvy převzaté z jiných publikací - je to vždy
  výslovně uvedeno. Tyto nápěvy vesměs patří k prastarému pokladu
  gregoriánského chorálu a jsou tedy public domain.)
  
  \emph{Texty Denní modlitby církve} 
  jsou majetkem České biskupské konference. 
  Výše uvedenou licenci proto nelze vztahovat na antifonář jako celek -
  týká se pouze jeho hudební složky.

  Texty žalmů a kantik jsou převzaté z webu \url{http://ebreviar.cz},
  jehož tvůrcům za jejich práci tímto srdečně děkuji.
  Z textů je vypuštěno to, co se zdá být pro modlitbu v chóru
  nadbytečné nebo rušivé (číslování žalmů i podle Vulgáty,
  nadpisy žalmů a připojené citáty z Písma a Otců)
  a kde jsem shledal odchylky od vydání breviáře z r. 1994, jsou
  podle něj opraveny.
}

%
% Vyrobi prazdnou stranku
%

\newcommand{\prazdnaStranka}{
  \newpage \mbox{}
  \newpage
}

%
% Nasledujici prikazy vyrabeji hlavicku a paticku pro
% male materialy obsahem a razem podobne materialum
% z LilyPondu
%

\newcommand{\malyTitulek}{
  \begin{center}
    {\large \textbf{\nazev}}
  \end{center}
  \begin{flushright}
    \autor
  \end{flushright}
}

\newcommand{\malaTiraz}{
  datum: \dmyyyydate \today
  
  licence: Creative Commons Attribution-ShareAlike 3.0 Unported
  
  projekt: In adiutorium - noty k liturgii hodin (http://inadiutorium.xf.cz)
  
  sazba programy \LaTeX a LilyPond \lilypondbookVersion
}

% cesky format data:
\renewcommand{\dateseparator}{.}

% format data, kde je jen rok
\newdateformat{onlyyeardate}{%
\THEYEAR}


%
% Nasledujici prikazy slouzi strukturovani uvnitr textu antifonare
%

% tyden
\newcommand{\nadpisTyden}[2]{
  \phantomsection%
  \begin{center}
  {\LARGE \textsc{#1}}
  \end{center}
  \fancyhead[CE]{#2}
  \addcontentsline{toc}{section}{#2}
}

% den
\newcommand{\nadpisDen}[1]{
  \phantomsection%
  \begin{center}
  {\LARGE #1}
  \end{center}
  \fancyhead[CO]{#1}
  \addcontentsline{toc}{subsection}{#1}
}

% hodinka
\newcommand{\nadpisHodinka}[1]{
  \begin{center}
  \textbf{#1}
  \end{center}
  \nopagebreak
}

% invitatorium samozrejme neni samostatna hodinka,
% ale tim, ze muze byt soucasti dvou ruznych,
% se funkcne osamostatnilo a davame mu nadpis na urovni hodinky
\newcommand{\invitatorium}{
  \nadpisHodinka{Invitatorium}
}

\newcommand{\modlitbaSeCtenim}{
  \nadpisHodinka{Modlitba se čtením}
}

\newcommand{\ranniChvaly}{
  \nadpisHodinka{Ranní chvály}
}

\newcommand{\modlitbaUprostredDne}{
  \nadpisHodinka{Modlitba uprostřed dne}
}

\newcommand{\nespory}{
  \nadpisHodinka{Nešpory}
}

\newcommand{\nesporyI}{
  \nadpisHodinka{První nešpory}
}

\newcommand{\nesporyII}{
  \nadpisHodinka{Druhé nešpory}
}

\newcommand{\kompletar}{
  \nadpisHodinka{Kompletář}
}

% nadpis hodinky uzivany v indexu svatku svatecniho zaltare
\newcommand{\nadpisHodinkaVIndexu}[1]{
  \noindent \underline{#1:}
}

% Varianty tech samych prikazu pro index svatku:
\newcommand{\idxModlitbaSeCtenim}{
  \nadpisHodinkaVIndexu{MČ}
}

\newcommand{\idxVigilie}{
  \nadpisHodinkaVIndexu{VI}
}

\newcommand{\idxRanniChvaly}{
  \nadpisHodinkaVIndexu{RCH}
}

\newcommand{\idxModlitbaUprostredDne}{
  \nadpisHodinkaVIndexu{MU}
}

\newcommand{\idxNespory}{
  \nadpisHodinkaVIndexu{N}
}

\newcommand{\idxNesporyI}{
  \nadpisHodinkaVIndexu{1N}
}

\newcommand{\idxNesporyII}{
  \nadpisHodinkaVIndexu{2N}
}

\newcommand{\idxKompletar}{
  \nadpisHodinkaVIndexu{K}
}

\newenvironment{idxObsahHory}{}{}


% Cilem je, aby tento nadpis nad sebou nemel tolik mista
% a tak umoznoval "prilepeni" predchazejici rubriky.
\newcommand{\nadpisZalmuBezMezery}[1]{%

  % prostredi center si svevolne pridava
  % dodatecny vertikalni prostor - musime ho tedy umele ubrat:
  \vspace{-5mm}%
  \begin{center}%
    \textbf{#1}%
  \end{center}%
  \nopagebreak%
}

% obvykle napr. "zalm 94"
\newcommand{\nadpisZalmu}[1]{%
  \vspace{2mm}%
  \begin{center}%
    \textbf{#1}%
  \end{center}%
  \nopagebreak%
}

\newcommand{\titulusPsalmi}[1]{\nadpisZalmu{#1}}


% Napr. "hymnus", "responsorium", "zalmy"
\newcommand{\nadpisTypTextu}[1]{
  \begin{flushleft}
  \noindent
  \textsc{#1}
  \end{flushleft}
  \nopagebreak
}

% rubrikovany text - treba i pro vlozeni do nerubrikovaneho
\newcommand{\rubr}[1]{%
  \textit{#1}%
}

% rubrika - samostatna
\newcommand{\rubrika}[1]{
  \vspace{2mm}
  \rubr{#1}
  \vspace{2mm}
}

% mezera, kterou se oddeli rubrika od textu, ke kteremu se _nevztahuje_,
% aby byla vic prilepena k tomu, ke kteremu se vztahuje
\newcommand{\rubrikaMezera}{
  \vspace{4mm}
}

% rubrika - predchazejici
\newcommand{\rubrikaPred}[1]{
  \rubrikaMezera
  % \textit{#1}
  \rubr{#1}
  \vspace{1mm}
}

% rubrika - nasledujici
\newcommand{\rubrikaPo}[1]{
  \vspace{1mm}
  % \textit{#1}
  \rubr{#1}
  \rubrikaMezera
}

% mezera pred hvezdickou pulici vers zalmu muze byt v pripade potreby
% zcela zkomprimovana
\newcommand{\asterisk}{%
\nobreak\hskip \fontdimen1\font plus \fontdimen2\font minus \fontdimen1\font
* }

\newcommand{\flexa}{%
\nobreak\hskip \fontdimen1\font plus \fontdimen2\font minus \fontdimen1\font
\dag\mbox{} }

\newcommand{\znackaStrofaZalmu}{ 
  \rubr{--}
}

\newcommand{\doxologieZkratka}{%
  \nopagebreak%
  \hspace{\fill}\mbox{}\linebreak[0]\hspace*{\parindent}\mbox{Sláva Otci.}%
}

% horizontalni cara dlouha jako pul stranky siroky sloupec sloupec
% ve svazku antifonare
\newcommand{\oddelovac}{
\noindent \line(1,0){150}
}


% z logiky liturgickych knih by znacky pro versiky mely byt
% rubrikovane, ne tucne. Zkousel jsem to, to ale v okoli zanikaji
% a myslim ze je dobre je vyrazne odlisit od psalmodie, ktera je alespon
% v zaltari bez not obklopuje
\newcommand{\versikSamostatny}[2]{
 
  \vspace{3mm}
  \noindent \textbf{V.} #1\\
  \textbf{O.} #2
}
\newcommand{\versikCteni}[2]{
  \versikSamostatny{#1}{#2}
}
\newcommand{\versik}[3]{

  \vspace{1mm}
  \rubrika{#1}\\
  \noindent \textbf{V.} #2\\
  \textbf{O.} #3
}
\newcommand{\versikTercie}[2]{\versik{dopoledne:}{#1}{#2}}
\newcommand{\versikSexta}[2]{\versik{v poledne:}{#1}{#2}}
\newcommand{\versikNona}[2]{\versik{odpoledne:}{#1}{#2}}

%% prikazy predpokladane preprocesorem zalmu pslm.rb:

% vetsina notovanych svazku chce mit kazdy zalm ve zvlastnim
% dvousloupci; v tech ostatnich se to predefinuje podle potreby
\newenvironment{psalmus}{\begin{multicols}{2}}{\end{multicols}}

\newcommand{\flex}{\flexa}


\newcommand{\titulniStranka}{
  \pagestyle{empty} % bez hlavicky a bez cisla stranky
  \begin{titlepage}
    \begin{center}
      
      \vspace*{5cm}
      
      {\Large\textbf{\nazev}}
          
      \vspace{1cm}
      
      \autor
      
      \vfill
      In adiutorium
      
      \onlyyeardate \today
    \end{center}
  \end{titlepage}
}

\newcommand{\autor}{Jakub Pavlík}
\newcommand{\nazev}{
  {\Huge Manuál}
  \vspace{1cm}

  ke společnému slavení\\
  denní modlitby církve
}

\newcommand{\refStranka}[1]{
(str. \pageref{#1})
}

%%%%%%%%%%%%%%%%%%%%%%%%%%%%%%%%%%%%%%%%%%%%%%%%%%%%%%%%%%%%
\begin{document}

\titulniStranka
\prazdnaStranka

\section*{Úvod}

V současné době není slavení denní modlitby církve chórovým způsobem
příliš rozšířené, takže není pro každého snadné získat příslušné
dovednosti přirozeným způsobem, pozorováním a napodobováním.
Navíc od liturgické reformy zřejmě neexistuje autoritativní příručka,
která by způsob slavení určovala a mohla sloužit jak jako učební text,
tak jako reference v případě pochybností.

Toto dílko má za cíl posbírat z různých míst autoritativní pravidla,
a v otázkách, které nejsou autoritativně upravené, upozornit na možnosti.
Využívá, vedle směrnic církevních liturgických autorit, jak zkušenosti 
se slavením liturgie hodin v různých komunitách
a společenstvích dnes, tak, s přihlédnutím ke změnám, staré liturgické
příručky.
Chce být pomocí pro ty, kdo usilují společné slavení denní modlitby církve
smysluplně, prakticky a důstojně uspořádat.

\section{Liturgický prostor a čas}
\subsection{Uspořádání a vybavení liturgického prostoru}
\subsection{Otázky času slavení}

% \section{Liturgické úlohy}
\section{Liturgické úlohy}

\subsection{Člen chóru}

Všem, kdo v chóru liturgii hodin společně slaví, přísluší všechny úkony,
které nejsou vyhrazeny nositelům úřadu nebo zvláštní úlohy.
Jsou to především společné zpěvy
(hymnus, antifony, žalmy, kantika)
a pak odpovědi ve střídavých zpěvech (začátek hodinek, invitatorium,
responsorium, verš, responsoriální kantika).
Dále odpovídají na přímluvy/prosby a modlí se Modlitbu Páně.

\subsection{Předsedající}
\subsubsection{Svěcený služebník (biskup/kněz/jáhen)}
\subsubsection{Nesvěcený}

\subsection{Kantor}

Jednomu nebo více kantorům přísluší
začínat antifonu, žalm\footnote{Denní modlitba církve. Ranní chvály, modlitba uprostřed dne, nešpory, modlitba před spaním, Kostelní Vydří 1994, s. 7-21: Úvod, odst. 45.},
zpívat sólové části veršů, responsorií a invitatoria.

Zvláště tam, kde se denní modlitba církve nezpívá, kantoři často
nejsou ustanoveni a výše vypsané úkoly připadnou podle určitých pravidel
vždy někomu jinému.

\subsection{Lektor}

Lektor předčítá krátké čtení (\emph{lectio brevis}, \emph{capitulum})
a čtení modlitby se čtením (\emph{lectio}), buďto ze svého sedadla,
nebo z místa k tomu určeného, na které pro čtení přechází.

Pokud se úloha lektora svěřuje na delší dobu (den, týden), může být vhodné
svěřit jednu nebo obě lekce modlitby se čtením někomu jinému, protože jsou
dlouhé a změna předčítajícího osvěží jak lektora, tak celý chór.

\subsection{Turiferář}

Turiferář obsluhuje kadidelnici, pokud se využívá možnosti okuřování
při evangelním kantiku v ranních chválách a nešporách.\footnote{Denní modlitba církve. Ranní chvály, modlitba uprostřed dne, nešpory, modlitba před spaním, Kostelní Vydří 1994, s. 7-21: Úvod, odst. 46.}


% \section{Stavební kameny liturgie hodin}
\section{Stavební kameny liturgie hodin}

\subsection{Začátek hodinek}
\label{sec:zacatek}

Při začátku všech hodinek všichni stojí.\footnote{Denní modlitba církve. Ranní chvály, modlitba uprostřed dne, nešpory, modlitba před spaním, Kostelní Vydří 1994, s. 7-21: Úvod, odst. 48.}

\subsubsection{První modlitba dne (\emph{Pane, otevři mé rty})}

První modlitba dne začíná veršem (viz pravidla pro verše obecně,
str. \pageref{sec:vers}) ze žalmu 51
\uv{Pane, otevři mé rty / a má ústa tě budou chválit.}, po kterém
bezprostředně následuje žalm invitatoria.\footnote{Před liturgickou
reformou po něm následovalo \emph{Bože, pospěš mi na pomoc}.}

Při verši všichni se všichni znamenají křížem na ústech.\footnote{Denní modlitba církve. Ranní chvály, modlitba uprostřed dne, nešpory, modlitba před spaním, Kostelní Vydří 1994, s. 7-21: Úvod, odst. 51.}

\subsubsection{Ostatní hodinky (\emph{Bože, pospěš mi na pomoc})}

Všechny ostatní hodinky začínají veršem \uv{Bože, pospěš mi na pomoc. / Slyš naše volání.}
následovaným doxologií (viz oddíl věnovaný doxologii, str. \pageref{sec:doxologie}) 
a - mimo postní dobu - aleluja.

Při slovech \uv{Bože, pospěš mi na pomoc} se všichni \uv{znamenají křížem
od čela k prsům a od levého ramene k pravému.}\footnote{Denní modlitba církve. Ranní chvály, modlitba uprostřed dne, nešpory, modlitba před spaním, Kostelní Vydří 1994, s. 7-21: Úvod, odst. 51.}

Doxologii říkají všichni společně, bezprostředně po odpovědi na verš.

\subsection{Zakončení hodinek}
\label{sec:zakonceni}
\subsubsection{Velké hodinky (požehnání s propuštěním)}
\subsubsection{Malé hodinky (\emph{Dobrořečme Pánu})}
\subsubsection{Kompletář (\emph{Dej nám, Bože, pokojnou noc})}

\subsection{Hymnus}
\label{sec:hymnus}

\subsection{Psalmodie}
\label{sec:psalmodie}
\subsubsection{Antifona}
\subsubsection{Žalm / Kantikum}

\begin{quote}
Žalmy se zpívají nebo recitují podle různých způsobů osvědčených tradicí
či zkušeností buď bez přerušení (neboli \uv{souvisle}),
nebo střídavě po verších nebo strofách dvěma chóry či skupinami
shromáždění, nebo responsoriálním způsobem.\footnote{Denní modlitba církve. Ranní chvály, modlitba uprostřed dne, nešpory, modlitba před spaním, Kostelní Vydří 1994, s. 7-21: Úvod, odst. 14.}
\end{quote}

Zřejmě nejrozšířenější je ten způsob \emph{střídavého přednesu}, 
kdy se dvě skupiny střídají po verších. Verše po sobě bezprostředně následují a 
v půlce verše se udělá krátká pauza.

\emph{Střídavý přednes} je pro liturgii hodin velmi vhodný, protože dobře 
vyvažuje zapojení a odpočinek všech.

Je obvyklé, že první poloverš prvního verše začíná kantor a první skupina
se připojuje od druhého poloverše. Pokud se při žalmu sedí, kantor
na jeho začátek, který pronáší sám, zpravidla povstává.

\emph{Nepřerušená recitace} postrádá výše zmíněnou výhodu střídavého přednesu.
Nikde jsem se s ní nesetkal, s výjimkou jedné mnišské komunity, kde je
v matutinu v každém nokturnu zařazen jeden dlouhý žalm, který vždy předčítá
\emph{jeden} z bratří a komunita tiše naslouchá (a když je žalm velmi dlouhý, 
jednou nebo dvakrát na označených místech vkládá antifonu).

\emph{Responsoriální přednes}, spočívající v tom, že žalm přednáší kantor
a chór po každé strofě vkládá antifonu, se nezdá být pro liturgii hodin
příliš vhodný. Mnohem méně totiž zapojuje chór a je tak únavnější,
resp. je pro naslouchající náročnější než při výše pojednaných způsobech,
být duchem při modlitbě, zvláště, jsou-li žalmy delší.

\begin{quote}
Při delších žalmech je v žaltáři provedeno takové rozdělení, které žalmy
dělí tak, že nastiňují trojitou strukturu jednotlivých částí denní modlitby 
církve.
Na konci každého oddílu se přidává \emph{Sláva Otci}.
Je dovoleno buď podržet tradiční způsob, nebo mezi oddělené části téhož žalmu
vkládat chvíle ticha, nebo celý žalm přednášet bez přerušení s jeho první
antifonou. Pokud se žalm recituje s vložením antifony po každé strofě,
postačí připojit \emph{Sláva Otci} na konci celého žalmu.\footnote{Denní modlitba církve. Ranní chvály, modlitba uprostřed dne, nešpory, modlitba před spaním, Kostelní Vydří 1994, s. 7-21: Úvod, odst. 14.}
\end{quote}

Liturgická reforma upravila liturgii hodin tak, že psalmodie každé hodinky
s výjimkou kompletáře sestává ze tří textů - kratších žalmů a kantik, nebo
oddílů žalmu delšího.\footnote{Tato trojčlennost není nic pro liturgii hodin
esenciálního - je dán koncepčním rozhodnutím poslední reformy, předtím
byl jiný.}

Úvod k Denní modlitbě církve nezmiňuje specifickou situaci modlitby uprostřed 
dne, která má ve zvláštních liturgických dobách (\emph{\uv{quattuor tempora}})
pro tři žalmy jen jednu antifonu. Z uspořádání knih a ze starších breviářů,
kde některé hodinky měly jednu antifonu na více žalmů vždy, plyne,
že se tato antifona má zpívat před a po celém bloku, nikoli mezi žalmy,
a na konci každého žalmu se připojí doxologie. Ale vzhledem k tomu, že
zde pro současnou liturgii není žádná výslovná norma, je opakování antifony 
po každém žalmu také legitimní možností.

Žalmy se mohou zpívat vstoje (délka psalmodie v současném breviáři to
přinejmenším mladým a zdravým dobře umožňuje) nebo vsedě.\footnote{Denní modlitba církve. Ranní chvály, modlitba uprostřed dne, nešpory, modlitba před spaním, Kostelní Vydří 1994, s. 7-21: Úvod, odst. 50.}

\subsubsection{Doxologie}
\label{sec:doxologie}

Doxologií \emph{Sláva Otci} se uzavírají všechny žalmy (příp. jejich části -
o možnostech viz výše) a kantika, není-li v rubrikách uvedeno jinak. 
(Jediným případem, kdy rubriky breviáře toto \emph{jinak} využívají,
je kantikum Dan 3, 57-58, které má vlastní doxologii
zapadající do jeho celkového rázu.)

Doxologie ve své \uv{kanonické} podobě má dva verše a mohou se tak na ní
při střídavém přednesu podílet oba chóry, což je symbolicky hodnotné.

\begin{quote}
Sláva Otci i Synu * i Duchu svatému,\\
jako byla na počátku, i nyní i vždycky * a na věky věků. Amen.
\end{quote}

Doxologie bývá spojována s vnějším gestem úklony - podle zvyklostí různě 
hluboké: od ůklony hlavy až po poklonu do pasu.

Někde se, i když je žalm recitován vsedě, na doxologii vstává,
jinde se zůstává sedět. (Povstání umožňuje hlubokou úklonu, vsedě je
elegantně možná jen úklona hlavy.)

Je dosti obvyklé omezit úklonu jen na slova 
\uv{Sláva Otci i Synu * i Duchu svatému} a druhý verš doxologie říkat už
s rovnými zády.
Další zvyklostí je - když se při žalmu sedí - povstat už na druhý poloverš 
posledního verše žalmu, při prvním verši doxologie se klanět
a na druhý poloverš druhého verše doxologie si zase sednout:

\begin{quote}
Oslavujte Hospodina, neboť je dobrý, *\\
\textbf{(povstat)} jeho milosrdenství trvá navěky.\\

\textbf{(poklona)} Sláva Otci i Synu *\\
i Duchu svatému,\\

\textbf{(narovnat se)} jako byla na počátku, i nyní i vždycky *\\
\textbf{(posadit se)} a na věky věků. Amen.
\end{quote}


\subsubsection{\uv{Responsoriální} kantika ze Zj 19 a 1 Tim 3}
\subsubsection{Žalm invitatoria}

\subsection{Čtení}
\label{sec:cteni}
\subsubsection{Krátké čtení (\emph{lectio brevis}, \emph{capitulum})}
\subsubsection{Čtení v modlitbě se čtením (\emph{lectio})}
\subsubsection{Evangelium v modlitbě se čtením}

\subsection{Responsorium}
\label{sec:responsorium}

\subsection{Verš}
\label{sec:vers}
\subsubsection{Před velkou lekcí modlitby se čtením}
\subsubsection{Po krátkém čtení}

\subsection{Te Deum}

\subsection{Prosby / Přímluvy}
\label{sec:prosby}

\subsection{Modlitba Páně}
\label{sec:otcenas}

\subsection{Modlitba}
\label{sec:modlitba}

\subsection{Závěrečná mariánská antifona}
\label{sec:mariaantifona}


\section{Uspořádání denní modlitby církve}
\subsection{Uvedení do první modlitby dne (invitatorium)}
\subsection{Modlitba se čtením}
\subsection{Ranní chvály}
\subsection{Modlitba uprostřed dne}
\subsection{Nešpory}
\subsection{Kompletář}

\section{Liturgický rok}

\section{Možnosti uzpůsobení}

\end{document}
