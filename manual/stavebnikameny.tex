\section{Stavební kameny liturgie hodin}

\subsection{Začátek hodinek}
\label{sec:zacatek}

Při začátku všech hodinek všichni stojí.\footnote{Denní modlitba církve. Ranní chvály, modlitba uprostřed dne, nešpory, modlitba před spaním, Kostelní Vydří 1994, s. 7-21: Úvod, odst. 48.}

\subsubsection{První modlitba dne (\emph{Pane, otevři mé rty})}

První modlitba dne začíná veršem (viz pravidla pro verše obecně,
str. \pageref{sec:vers}) ze žalmu 51
\uv{Pane, otevři mé rty / a má ústa tě budou chválit.}, po kterém
bezprostředně následuje žalm invitatoria.\footnote{Před liturgickou
reformou po něm následovalo \emph{Bože, pospěš mi na pomoc}.}

Při verši všichni se všichni znamenají křížem na ústech.\footnote{Denní modlitba církve. Ranní chvály, modlitba uprostřed dne, nešpory, modlitba před spaním, Kostelní Vydří 1994, s. 7-21: Úvod, odst. 51.}

\subsubsection{Ostatní hodinky (\emph{Bože, pospěš mi na pomoc})}

Všechny ostatní hodinky začínají veršem \uv{Bože, pospěš mi na pomoc. / Slyš naše volání.}
následovaným doxologií (viz oddíl věnovaný doxologii, str. \pageref{sec:doxologie}) 
a - mimo postní dobu - aleluja.

Při slovech \uv{Bože, pospěš mi na pomoc} se všichni \uv{znamenají křížem
od čela k prsům a od levého ramene k pravému.}\footnote{Denní modlitba církve. Ranní chvály, modlitba uprostřed dne, nešpory, modlitba před spaním, Kostelní Vydří 1994, s. 7-21: Úvod, odst. 51.}

Doxologii říkají všichni společně, bezprostředně po odpovědi na verš.

\subsection{Zakončení hodinek}
\label{sec:zakonceni}
\subsubsection{Velké hodinky (požehnání s propuštěním)}
\subsubsection{Malé hodinky (\emph{Dobrořečme Pánu})}
\subsubsection{Kompletář (\emph{Dej nám, Bože, pokojnou noc})}

\subsection{Hymnus}
\label{sec:hymnus}

\subsection{Psalmodie}
\label{sec:psalmodie}
\subsubsection{Antifona}
\subsubsection{Žalm / Kantikum}

\begin{quote}
Žalmy se zpívají nebo recitují podle různých způsobů osvědčených tradicí
či zkušeností buď bez přerušení (neboli \uv{souvisle}),
nebo střídavě po verších nebo strofách dvěma chóry či skupinami
shromáždění, nebo responsoriálním způsobem.\footnote{Denní modlitba církve. Ranní chvály, modlitba uprostřed dne, nešpory, modlitba před spaním, Kostelní Vydří 1994, s. 7-21: Úvod, odst. 14.}
\end{quote}

Zřejmě nejrozšířenější je ten způsob \emph{střídavého přednesu}, 
kdy se dvě skupiny střídají po verších. Verše po sobě bezprostředně následují a 
v půlce verše se udělá krátká pauza.

\emph{Střídavý přednes} je pro liturgii hodin velmi vhodný, protože dobře 
vyvažuje zapojení a odpočinek všech.

Je obvyklé, že první poloverš prvního verše začíná kantor a první skupina
se připojuje od druhého poloverše. Pokud se při žalmu sedí, kantor
na jeho začátek, který pronáší sám, zpravidla povstává.

\emph{Nepřerušená recitace} postrádá výše zmíněnou výhodu střídavého přednesu.
Nikde jsem se s ní nesetkal, s výjimkou jedné mnišské komunity, kde je
v matutinu v každém nokturnu zařazen jeden dlouhý žalm, který vždy předčítá
\emph{jeden} z bratří a komunita tiše naslouchá (a když je žalm velmi dlouhý, 
jednou nebo dvakrát na označených místech vkládá antifonu).

\emph{Responsoriální přednes}, spočívající v tom, že žalm přednáší kantor
a chór po každé strofě vkládá antifonu, se nezdá být pro liturgii hodin
příliš vhodný. Mnohem méně totiž zapojuje chór a je tak únavnější,
resp. je pro naslouchající náročnější než při výše pojednaných způsobech,
být duchem při modlitbě, zvláště, jsou-li žalmy delší.

\begin{quote}
Při delších žalmech je v žaltáři provedeno takové rozdělení, které žalmy
dělí tak, že nastiňují trojitou strukturu jednotlivých částí denní modlitby 
církve.
Na konci každého oddílu se přidává \emph{Sláva Otci}.
Je dovoleno buď podržet tradiční způsob, nebo mezi oddělené části téhož žalmu
vkládat chvíle ticha, nebo celý žalm přednášet bez přerušení s jeho první
antifonou. Pokud se žalm recituje s vložením antifony po každé strofě,
postačí připojit \emph{Sláva Otci} na konci celého žalmu.\footnote{Denní modlitba církve. Ranní chvály, modlitba uprostřed dne, nešpory, modlitba před spaním, Kostelní Vydří 1994, s. 7-21: Úvod, odst. 14.}
\end{quote}

Liturgická reforma upravila liturgii hodin tak, že psalmodie každé hodinky
s výjimkou kompletáře sestává ze tří textů - kratších žalmů a kantik, nebo
oddílů žalmu delšího.\footnote{Tato trojčlennost není nic pro liturgii hodin
esenciálního - je dán koncepčním rozhodnutím poslední reformy, předtím
byl jiný.}

Úvod k Denní modlitbě církve nezmiňuje specifickou situaci modlitby uprostřed 
dne, která má ve zvláštních liturgických dobách (\emph{\uv{quattuor tempora}})
pro tři žalmy jen jednu antifonu. Z uspořádání knih a ze starších breviářů,
kde některé hodinky měly jednu antifonu na více žalmů vždy, plyne,
že se tato antifona má zpívat před a po celém bloku, nikoli mezi žalmy,
a na konci každého žalmu se připojí doxologie. Ale vzhledem k tomu, že
zde pro současnou liturgii není žádná výslovná norma, je opakování antifony 
po každém žalmu také legitimní možností.

Žalmy se mohou zpívat vstoje (délka psalmodie v současném breviáři to
přinejmenším mladým a zdravým dobře umožňuje) nebo vsedě.\footnote{Denní modlitba církve. Ranní chvály, modlitba uprostřed dne, nešpory, modlitba před spaním, Kostelní Vydří 1994, s. 7-21: Úvod, odst. 50.}

\subsubsection{Doxologie}
\label{sec:doxologie}

Doxologií \emph{Sláva Otci} se uzavírají všechny žalmy (příp. jejich části -
o možnostech viz výše) a kantika, není-li v rubrikách uvedeno jinak. 
(Jediným případem, kdy rubriky breviáře toto \emph{jinak} využívají,
je kantikum Dan 3, 57-58, které má vlastní doxologii
zapadající do jeho celkového rázu.)

Doxologie ve své \uv{kanonické} podobě má dva verše a mohou se tak na ní
při střídavém přednesu podílet oba chóry, což je symbolicky hodnotné.

\begin{quote}
Sláva Otci i Synu * i Duchu svatému,\\
jako byla na počátku, i nyní i vždycky * a na věky věků. Amen.
\end{quote}

Doxologie bývá spojována s vnějším gestem úklony - podle zvyklostí různě 
hluboké: od ůklony hlavy až po poklonu do pasu.

Někde se, i když je žalm recitován vsedě, na doxologii vstává,
jinde se zůstává sedět. (Povstání umožňuje hlubokou úklonu, vsedě je
elegantně možná jen úklona hlavy.)

Je dosti obvyklé omezit úklonu jen na slova 
\uv{Sláva Otci i Synu * i Duchu svatému} a druhý verš doxologie říkat už
s rovnými zády.
Další zvyklostí je - když se při žalmu sedí - povstat už na druhý poloverš 
posledního verše žalmu, při prvním verši doxologie se klanět
a na druhý poloverš druhého verše doxologie si zase sednout:

\begin{quote}
Oslavujte Hospodina, neboť je dobrý, *\\
\textbf{(povstat)} jeho milosrdenství trvá navěky.\\

\textbf{(poklona)} Sláva Otci i Synu *\\
i Duchu svatému,\\

\textbf{(narovnat se)} jako byla na počátku, i nyní i vždycky *\\
\textbf{(posadit se)} a na věky věků. Amen.
\end{quote}


\subsubsection{\uv{Responsoriální} kantika ze Zj 19 a 1 Tim 3}
\subsubsection{Žalm invitatoria}

\subsection{Čtení}
\label{sec:cteni}
\subsubsection{Krátké čtení (\emph{lectio brevis}, \emph{capitulum})}
\subsubsection{Čtení v modlitbě se čtením (\emph{lectio})}
\subsubsection{Evangelium v modlitbě se čtením}

\subsection{Responsorium}
\label{sec:responsorium}

\subsection{Verš}
\label{sec:vers}
\subsubsection{Před velkou lekcí modlitby se čtením}
\subsubsection{Po krátkém čtení}

\subsection{Te Deum}

\subsection{Prosby / Přímluvy}
\label{sec:prosby}

\subsection{Modlitba Páně}
\label{sec:otcenas}

\subsection{Modlitba}
\label{sec:modlitba}

\subsection{Závěrečná mariánská antifona}
\label{sec:mariaantifona}
