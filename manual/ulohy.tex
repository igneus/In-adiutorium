\section{Liturgické úlohy}

\subsection{Člen chóru}

Všem, kdo v chóru liturgii hodin společně slaví, přísluší všechny úkony,
které nejsou vyhrazeny nositelům úřadu nebo zvláštní úlohy.
Jsou to především společné zpěvy 
(hymnus, antifony, žalmy, kantika)
a pak odpovědi ve střídavých zpěvech (začátek hodinek, invitatorium,
responsorium, verš, responsoriální kantika).
Dále odpovídají na přímluvy/prosby a modlí se Modlitbu Páně.

\subsection{Předsedající}
\subsubsection{Svěcený služebník (biskup/kněz/jáhen)}
\subsubsection{Nesvěcený}

\subsection{Kantor}

Jednomu nebo více kantorům přísluší 
začínat antifonu, žalm\footnote{Denní modlitba církve. Ranní chvály, modlitba uprostřed dne, nešpory, modlitba před spaním, Kostelní Vydří 1994, s. 7-21: Úvod, odst. 45.},
zpívat sólové části veršů, responsorií a invitatoria.

Zvláště tam, kde se denní modlitba církve nezpívá, kantoři často
nejsou ustanoveni a výše vypsané úkoly připadnou podle určitých pravidel
vždy někomu jinému. 

\subsection{Lektor}

Lektor předčítá krátké čtení (\emph{lectio brevis}, \emph{capitulum}) 
a čtení modlitby se čtením (\emph{lectio}), buďto ze svého sedadla,
nebo z místa k tomu určeného, na které pro čtení přechází.

Pokud se úloha lektora svěřuje na delší dobu (den, týden), může být vhodné
svěřit jednu nebo obě lekce modlitby se čtením někomu jinému, protože jsou
dlouhé a změna předčítajícího osvěží jak lektora, tak celý chór.

\subsection{Turiferář}

Turiferář obsluhuje kadidelnici, pokud se využívá možnosti okuřování
při evangelním kantiku v ranních chválách a nešporách.\footnote{Denní modlitba církve. Ranní chvály, modlitba uprostřed dne, nešpory, modlitba před spaním, Kostelní Vydří 1994, s. 7-21: Úvod, odst. 46.}
