\section{Liturgické úlohy}

% !!! Závěrečná modlitba: předsedající, nebo hebdomadář?
% V Novém Dvoře hebdomadář, i nekněz. Je to mnišská zvláštnost?

\subsection{Člen chóru}

Všem, kdo v chóru liturgii hodin společně slaví, přísluší všechny úkony,
které nejsou vyhrazeny nositelům úřadu nebo zvláštní úlohy.
Jsou to především společné zpěvy 
(hymnus, antifony, žalmy, kantika)
a pak odpovědi ve střídavých zpěvech (začátek hodinek, invitatorium,
responsorium, verš, responsoriální kantika).
Dále odpovídají na přímluvy/prosby a modlí se Modlitbu Páně.

\subsection{Předsedající}
\subsection{Kantor}
\subsection{Lektor}

Lektor předčítá krátké čtení (\emph{lectio brevis}, \emph{capitulum}) 
a čtení modlitby se čtením (\emph{lectio}), buďto ze svého sedadla,
nebo z místa k tomu určeného, na které pro čtení přechází.

Pokud se úloha lektora svěřuje na delší dobu (den, týden), může být vhodné
svěřit jednu nebo obě lekce modlitby se čtením někomu jinému, protože jsou
dlouhé a změna předčítajícího osvěží jak lektora, tak celý chór.

\subsection{Turiferář}

